\documentclass[12pt]{article}

\usepackage{amssymb, amsmath, amsfonts}
\usepackage{moreverb}
\usepackage{graphicx}
\usepackage{enumerate}
\usepackage[margin=0.75in]{geometry}
\usepackage{graphics}
\usepackage{color}
\usepackage{array}
\usepackage{float}
\usepackage{hyperref}
\usepackage{textcomp}
\usepackage{bbold}
\usepackage{alltt}
\usepackage{physics}
\usepackage{mathtools}
\usepackage{amsthm}
\usepackage{tikz}
\usetikzlibrary{positioning}
\usetikzlibrary{arrows}
\usepackage{pgfplots}
\usepackage{bigints}
\allowdisplaybreaks
\pgfplotsset{compat=1.12}

\theoremstyle{plain}
\newtheorem*{theorem*}{Theorem}
\newtheorem{theorem}{Theorem}
\newtheorem*{lemma*}{Lemma}
\newtheorem{lemma}{Lemma}

\newenvironment{definition}[1][Definition]{\begin{trivlist}
\item[\hskip \labelsep {\bfseries #1}]}{\end{trivlist}}

\title{\bf HW \#1}
\author{\bf Sam Fleischer}
\date{\bf January 15, 2015}

\pgfplotsset{compat=1.12}

\begin{document}
\textbf{MATH 201B \hfill Analysis \ \ \hfill Winter 2016\ \ \ }

{\let\newpage\relax\maketitle}

\section*{Exercise 1.1}
\textit{Complete the proof of the the Monotone Class Theorem.}

\begin{lemma}
    \label{monotone_class_lemma}
    The arbitrary intersection of monotone classes in a monotone class.
\end{lemma}
\begin{proof}
    Let $\mathcal{S}$ be the arbitrary intersection of monotone classes $M_j$ for $j \in J$, where $J$ is an index set.  Then let $S_1 \subset S_2 \subset S_3 \subset \dots$ and $S_i \in \mathcal{S}\ \forall i = 1, 2, \dots$.  Then since each $S_i \in M_j$ for each $M_j$ and each $M_j$ is a monotone class, then $\bigcup_{i=1}^\infty S_i \in M_j$ for each $M_j$.  Thus $\bigcup_{i=1}^\infty S_i \in \mathcal{S}$.  Now let $S_1 \supset S_2 \supset S_3 \supset \dots$ and $S_i \in \mathcal{S}\ \forall i = 1, 2, \dots$.  Then since $S_i in M_j$ for each $M_j$ and each $M_j$ is a monotone class, then $\bigcap_{i=1}^\infty S_i \in M_j$ for each $M_j$.  Thus $\bigcap_{i=1}^\infty S_i \in \mathcal{S}$.  Thus $\mathcal{S}$ is a monotone class.
\end{proof}

\begin{theorem}[Monotone Class Theorem]
    Let $\Omega$ be a set a let $\mathcal{A}$ be an algebra of subsets of $\Omega$ such that $\Omega, \emptyset \in \mathcal{A}$.  Then there exists a smallest monotone class $\mathcal{S}$ that contain $\mathcal{A}$.  That class, $\mathcal{S}$, is also the smallest sigma-algebra that contains $\mathcal{A}$.
\end{theorem}

\begin{proof}
    Let $S$ be the intersection of all monotone classes $M_i$ that contain $\mathcal{A}$.  By Lemma \ref{monotone_class_lemma}, $S$ is a monotone class, and thus the smallest monotone class containing $\mathcal{A}$.

    Pick $A \in \mathcal{A}$ and construct $C(A) = \{B \in\mathcal{ S}\ |\ B \cup A \in\mathcal{ S}\}$.  By construction, $C(A) \subset\mathcal{ S}$.  Since $\mathcal{A}$ is an algebra, $\mathcal{A}$ is closed under finite unions, and thus $\mathcal{A} \subset C(A)$.  Now we show $C(A)$ is a monotone class, which would show $\mathcal{S} \subset C(A)$, implying $C(A) = \mathcal{S}$.  Take $B_1 \subset B_2 \subset B_3 \subset \dots$ and $B_i \in C(A)\ \forall i = 1, 2, \dots$.  Then $B_i \cup A \in \mathcal{S}\ \forall i = 1, 2, \dots$ and $(B_1 \cup A) \subset (B_2 \cup A) \subset \dots$.  Then since $\mathcal{S}$ is a monotone class, $\bigcup_{i=1}^\infty (B_i \cup A) \in \mathcal{S}$, but $\bigcup_{i=1}^\infty (B_i \cup A) = \qty(\bigcup_{i=1}^\infty B_i) \cup A$.  Thus $\bigcup_{i=1}^\infty B_i \in C(A)$.  Similarly, take $D_1 \supset D_2 \supset \dots$ and $D_i \in C(A)\ \forall i = 1, 2, \dots$.  Then $D_i \cup A \in \mathcal{S}\ \forall i = 1, 2, \dots$ and $(D_1 \cup A) \supset (D_2 \cup A) \supset \dots$.  Then since $\mathcal{S}$ is a monotone class, $\bigcap_{i=1}^\infty (D_i \cup A) \in \mathcal{S}$, but $\bigcap_{i=1}^\infty (D_i \cup A) = \qty(\bigcap_{i=1}^\infty D_i) \cup A$.  Thus $\bigcap_{i=1}^\infty D_i \in C(A)$.  This proves that $C(A)$ is a monotone class, and thus $C(A) = \mathcal{S}$.

    Now we extend the definition of $C(A)$ to be defined for any $A \in \mathcal{S}$.  Pick $A' \in \mathcal{S}$.  Then since $A' \in C(A)\ \forall A \in \mathcal{A}$, then $A \in C(A')\ \forall A' \in \mathcal{A}$.  Thus $\mathcal{A} \subset C(A')$.  Now we show $C(A')$ is a monotone class, which would show $\mathcal{S} \subset C(A')$, implying $C(A') = \mathcal{S}$.  Take $B_1 \subset B_2 \subset B_3 \subset \dots$ and $B_i \in C(A')\ \forall i = 1, 2, \dots$.  Then $B_i \cup A' \in \mathcal{S}\ \forall i = 1, 2, \dots$ and $(B_1 \cup A') \subset (B_2 \cup A') \subset \dots$.  Then since $\mathcal{S}$ is a monotone class, $\bigcup_{i=1}^\infty (B_i \cup A') \in \mathcal{S}$, but $\bigcup_{i=1}^\infty (B_i \cup A') = \qty(\bigcup_{i=1}^\infty B_i) \cup A'$.  Thus $\bigcup_{i=1}^\infty B_i \in C(A')$.  Similarly, take $D_1 \supset D_2 \supset \dots$ and $D_i \in C(A')\ \forall i = 1, 2, \dots$.  Then $D_i \cup A' \in \mathcal{S}\ \forall i = 1, 2, \dots$ and $(D_1 \cup A') \supset (D_2 \cup A') \supset \dots$.  Then since $\mathcal{S}$ is a monotone class, $\bigcap_{i=1}^\infty (D_i \cup A') \in \mathcal{S}$, but $\bigcap_{i=1}^\infty (D_i \cup A') = \qty(\bigcap_{i=1}^\infty D_i) \cup A'$.  Thus $\bigcap_{i=1}^\infty D_i \in C(A')$.  This proves that $C(A')$ is a monotone class, and thus $C(A') = \mathcal{S}$.  Thus $\mathcal{S}$ is closed under finite unions.

    Now define $C = \{B \in \mathcal{S}\ |\ B^C \in \mathcal{S}\}$.  Since $\mathcal{A}$ is an algebra, $\mathcal{A}$ is closed under complimentation, and thus $\mathcal{A} \subset C$.  Now take $B_1 \subset B_2 \subset \dots$ and $B_i \in C\ \forall i = 1, 2, \dots$.  Then since $B_1^C \supset B_2^C \supset \dots$ and $B_i^C \in \mathcal{S}\ \forall i = 1, 2, \dots$, and since $\mathcal{S}$ is a monotone class, then $\bigcap_{i=1}^\infty \qty(B_i^C) \in S$.  However, $\bigcap_{i=1}^\infty \qty(B_i^C) = \qty(\bigcup_{i=1}^\infty B_i)^C$, and thus $\bigcup_{i=1}^\infty B_i \in C$.  Then take $D_1 \supset D_2 \supset \dots$ and $D_i \in C\ \forall i = 1, 2, \dots$.  Then since $D_1^C \subset D_2^C \subset \dots$ and $D_i^C \in \mathcal{S}$, and since $\mathcal{S}$ is a monotone class, then $\bigcup_{i=1}^\infty \qty(D_1^C) \in S$.  However, $\bigcup_{i=1}^\infty \qty(D_i^C) = \qty(\bigcap_{i=1}^\infty D_C)^C$, and thus $\bigcap_{i=1}^\infty D_i \in C$.  Thus $C$ is a monotone class containing $\mathcal{A}$, and thus $\mathcal{S} \subset C$, proving $C = \mathcal{S}$.  Thus $\mathcal{S}$ is closed under complementation.

    Now we show $\mathcal{S}$ is closed under countable unions and intersections.  Consider a sequene of sets $\{A_i\}_{i=1}^\infty \in \mathcal{S}$.  Then form $B_n = \bigcup_{i=1}^n$.  Since each $B_n$ is a finite union of elements in $\mathcal{S}$, then each $B_n \in \mathcal{S}$.  Also, $B_1 \subset B_2 \subset \dots$.  Since $\mathcal{S}$ is a monotone class, $\bigcup_{n=1}^\infty B_n \in \mathcal{S}$, but $\bigcup_{i=1}^\infty A_i = \bigcup_{n=1}^\infty B_n$, and thus $\mathcal{S}$ is closed under countable unions.  Similarly, form $D_n = \bigcup_{i=1}^n A_i^C$.  Since each $D_i$ is a finite union of elements in $\mathcal{S}$ ($\mathcal{S}$ is closed under complementation), then each $D_i \in \mathcal{S}$.  Also, $D_1 \subset D_2 \subset \dots$.  Since $\mathcal{S}$ is a monotone class, $\bigcup_{n=1}^\infty D_n \in S$, but $\bigcup_{n=1}^\infty D_n = \bigcup_{i=1}^\infty \qty(A_i^C) = \qty(\bigcap_{i=1}^\infty A_i)^C$.  Again, since $\mathcal{S}$ is closed under complementation, $\bigcap_{i=1}^\infty A_i \in \mathcal{S}$.  Thus $\mathcal{S}$ is closed under countable unions and intersections.

    This proves $\mathcal{S}$ is a $\sigma$-algebra.  However, every $\sigma$-algebra is a monotone class, and thus $\mathcal{S}$ must be the smallest $\sigma$-algebra containing $\mathcal{A}$ since it is defined as the smallest monotone class containing $\mathcal{A}$.
\end{proof}

\section*{Exercise 1.2}
\textit{With regard to the remark about continuous functions in Section 1.5, show that $f$ is continuous (in the sense of the usual $\varepsilon, \delta$ definition) if and only if $f$ is both upper and lower semicontinuous.  Show that $f$ is upper semicontinuous at $x$ if and only if, for every sequence $x_1, x_2, \dots$ converging to $x$, we have $f(x) \geq \overline{\lim}_{n\rightarrow \infty} f(x_n)$.}

\begin{definition}
    Consider $f\ :\ \Omega \rightarrow \mathbb{R}$, and define $L_f(t) = \{x \in \Omega\ |\ f(x) > t\}$ and $U_f(t) = \{x \in \Omega\ |\ f(x) < t\}$.  Then $f$ is \emph{lower semicontinuous} if $L_f(t)$ is open $\forall t \in \mathbb{R}$ and $f$ is \emph{upper semicontinous} if $U_f(t)$ is open $\forall t \in \mathbb{R}$.
\end{definition}
\begin{theorem}
    Let $f\ :\ \Omega \rightarrow \mathbb{R}$.  Then $f$ is continuous if and only if $f$ is both upper and lower semincontinuous.
\end{theorem}
\begin{proof}
    Let $f$ a continuous function.  Then $\forall x \in \Omega$ and $\varepsilon > 0$, $\exists \delta > 0$ such that $f(B_\delta(x)) \subset B_\varepsilon(f(x))$.  Fix $t \in \mathbb{R}$ and let $x_L \in L_f(t)$.  Then $f(x_L) = t + \ell$ for some $\ell > 0$.  Now take $\varepsilon = \ell$.  Then by the continuity of $f$, $\exists \delta_\ell$ such that $f(B_{\delta_\ell}(x_L)) \subset B_\ell(t + \ell)$.  But since $t_0 \in B_\ell(t + \ell) \implies t_0 > t$, then $B_{\delta_\ell}(x_L) \subset L_f(t)$.  Thus $L_f(t)$ is open.  Now let $x_U \in U_f(t)$.  Then $f(x_U) = t - u$ for some $u > 0$.  Again, take $\varepsilon = u$, and again by the continuity of $f$, $\exists \delta_u$ such that $f(B_{\delta_u}(x_U)) \subset B_u(f(x_U))$.  But since $t_0 \in B_u(t - u) \implies t_0 < t$, then $B_{\delta_u}(x_U) \subset U_f(t)$.  Thus $U_f(t)$ is open.  Thus $f$ is both upper and lower semicontinuous.

    Now let $f$ be both upper and lower semicontinuous.  Thus $\forall t \in \mathbb{R}$, $L_f(t)$ and $U_f(t)$ are open.  Then pick $x \in \Omega$ and let $t = f(x)$.  Choose $\varepsilon > 0$ and let $t_1 = t - \varepsilon$ and $t_2 = t + \varepsilon$.  Then $x \in L_f(t_2)$ and $x \in U_f(t_1)$.  Since $L_f(t_2)$ and $U_f(t_1)$ are open, then $\exists \delta_L$ and $\delta_U$ such that $f(B_{\delta_L}(x)) \subset B_\varepsilon(t_2)$ and $f(B_{\delta_U}(x)) \subset B_\varepsilon(t_1)$.  Choose $\delta = \min(\delta_L, \delta_U)$ and let $x_0 \in B_\delta(x)$.  Then $t_1 < f(x_0) < t_2$, which shows $f(x_0) \in B_\varepsilon(t)$, and thus $f$ is continuous.

    Thus $f$ is continuous if and only if $f$ is both lower and upper semicontinuous.
\end{proof}

\begin{theorem}
    Let $f\ :\ \Omega \rightarrow \mathbb{R}$.  Then $f$ is upper semicontinuous at $x$ if and only if for every sequence $\{x_i\} \in \Omega$ such that $x_i \rightarrow x$, we have $f(x) \geq \overline{\lim}_{i\rightarrow \infty}f(x_i)$.
\end{theorem}
\begin{proof}
    Fix $x \in \Omega$ and let $f$ by upper semicontinuous at $x$.  Then $\forall \varepsilon > 0$, $\exists \delta > 0$ such that $d_\Omega(x, y) < \delta \implies f(y) - f(x) < \varepsilon$, i.e.~$f(y) < f(x) + \varepsilon$.  Now consider a sequence $\{x_i\} \in \Omega$ such that $x_i \rightarrow x$.  Then consider a sequence $\varepsilon_k$ such that $\varepsilon_k \rightarrow 0$.  By the upper semicontinuity of $f$, $\exists I_k$ such that $\sup_{i \geq I_k}\{f(x_i)\} \leq f(x) + \varepsilon_k$ for each $k = 1, 2, \dots$.  Form a sequence $L_k = \max\{k, I_k\}$ and note $L_k \rightarrow \infty$, $L_k \geq I_k$, and $L_k \geq k$ for each $k = 1, 2, \dots$.  Then
    \begin{align*}
        \sup_{i \geq L_k}\{f(x_i)\} &\leq \sup_{i \geq I_k}\{f(x_i)\} \leq f(x) + \varepsilon_k \\
        \implies \lim_{k \rightarrow \infty}\qty(\sup_{i \geq L_k}\{f(x_i)\}) &\leq \lim_{k \rightarrow \infty} \qty(f(x) + \varepsilon_k) \\
        \implies \lim_{k \rightarrow \infty}\qty(\sup_{i \geq k}\{f(x_i)\}) &\leq f(x) + 0 = f(x) \\
        \implies \overline{\lim}_{i \rightarrow \infty} f(x_i) &\leq f(x)
    \end{align*}
\end{proof}

\section*{Exercise 1.3}
\textit{Prove the assertion made in Section 1.5 that for any Borel set $A \subset \mathbb{R}$ and any $\sigma$-algebra $\Sigma$ the set $\{x\ |\ f(x) \in A\} = f^{-1}(A)$ is $\Sigma$-measurable whenever the function $f$ is $\Sigma$-measurable.}

\begin{definition}
    Consider $f\ :\ \Omega \rightarrow \mathbb{R}$ and let $\Sigma$ be a $\sigma$-algebra on $\Omega$.  We say that $f$ is a \emph{measurable function} (with respect to $\Sigma$) if the set $L_f(t) = \{x \in \Omega\ |\ f(x) > t\} = f^{-1}((t, \infty))$ is measurable, i.e.~$L_f(t) \in \Sigma$, for every $t \in \mathbb{R}$.
\end{definition}
\begin{theorem}
    Let $f\ :\ \Omega \rightarrow \mathbb{R}$.  Let $\mathcal{B}$ be the Borel $\sigma$-algebra on $\mathbb{R}$ and let $A \in \mathcal{B}$.  Then the set $P_A = \{x\ |\ f(x) \in A\} = f^{-1}(A)$ is $\Sigma$-measurable whenever $f$ is $\Sigma$-measurable.
\end{theorem}
\begin{proof}
    Consider the interval $(a, b) \subset \mathbb{R}$ ($a < b$).  Note that
    \begin{align*}
        (a, b) &= (-\infty, b) \cap (a, \infty) \\
        &= [b, \infty)^C \cap (a, \infty) \\
        &= \qty[\bigcap_{i = 1}^\infty(b - 2^{-i}, \infty)]^C \cap (a, \infty)
    \end{align*}
    This shows that any open ball in $\mathbb{R}$ is countably $\sigma$-algebraicly generated from sets of the form $(t, \infty)$.  Due to the properties of preimages,
    \begin{align*}
        f^{-1}((a,b)) &= f^{-1}\qty(\qty[\bigcap_{i=1}^\infty (b - 2^{-i}, \infty)]^C \cap (a, \infty)) \\
        &= \qty[\bigcap_{i=1}^\infty f^{-1}(b - 2^{-i}, \infty)]^C \cap f^{-1}((a, \infty)) \\
        &= \qty[\bigcap_{i=1}^\infty L_f(b - 2^{-i})]^C \cap L_f(a)
    \end{align*}
    Thus, since $\Sigma$ is closed under countable unions, countable intersections, and complements, $f^{-1}((a,b)) \in \Sigma$.

    Now consider $A \in \mathcal{B}$.  Then since $\mathcal{B}$ is the smallest $\sigma$-algebra containing the open balls, then $A$ is countably $\sigma$-algebraicly generated from open balls.  However, preimages are closed under arbitrary unions, arbitrary intersections, and complements, and thus $f^{-1}(A) = P_A \in \Sigma$.  Thus $P_A$ is $\Sigma$-measurable whenever $f$ is $\Sigma$-measurable.
\end{proof}

\end{document}
