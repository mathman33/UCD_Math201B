\documentclass[12pt]{article}

\usepackage{amssymb, amsmath, amsfonts}
\usepackage{moreverb}
\usepackage{graphicx}
\usepackage{enumerate}
\usepackage[margin=0.75in]{geometry}
\usepackage{graphics}
\usepackage{color}
\usepackage{array}
\usepackage{float}
\usepackage{hyperref}
\usepackage[makeroom]{cancel}
\usepackage{textcomp}
\usepackage{bbold}
\usepackage{alltt}
\usepackage{physics}
\usepackage{mathtools}
\usepackage{amsthm}
\usepackage{tikz}
\usetikzlibrary{positioning}
\usetikzlibrary{arrows}
\usepackage{pgfplots}
\usepackage{bigints}
\allowdisplaybreaks
\pgfplotsset{compat=1.12}

\newcommand{\E}{\varepsilon}

\theoremstyle{plain}
\newtheorem*{theorem*}{Theorem}
\newtheorem{theorem}{Theorem}
\newtheorem*{lemma*}{Lemma}
\newtheorem{lemma}{Lemma}

\newenvironment{definition}[1][Definition]{\begin{trivlist}
\item[\hskip \labelsep {\bfseries #1}]}{\end{trivlist}}

\title{\bf HW \#2}
\author{\bf Sam Fleischer}
\date{\bf January 29, 2015}

\pgfplotsset{compat=1.12}

\begin{document}
\textbf{MATH 201B \hfill Analysis \ \ \hfill Winter 2016\ \ \ }

{\let\newpage\relax\maketitle}

\section*{Hunter and Nachtergaele 7.1}
\emph{Let $\phi_n$ be the functions defined in (7.7) $$\phi_n(x) = c_n(1 + \cos x)^n$$ where $c_n$ is chosen such that $$\int_\mathbb{T}\phi_n(x) \dd x = 1$$ for all $n$.}
\begin{enumerate}[\bf (a)]
    \item
        \emph{Prove (7.5).  $$\lim_{n\rightarrow \infty}\int_{\delta\leq|x|\leq \pi}\phi_n(x) \dd x = 0$$ for every $\delta > 0$.} \\

        Let $\delta > 0$ and for ease, define $\mathbb{D} = [-\pi, -\delta]\cup[\delta, \pi]$.  
        \begin{align*}
            \int_\mathbb{D} \phi_n(x)\dd x = \frac{\int_\mathbb{D} (1 + \cos x)^n \dd x}{\int_\mathbb{T} (1 + \cos x)^n\dd x}
        \end{align*}
        since
        \begin{align*}
            c_n = \frac{1}{\int_\mathbb{T}(1 + \cos x)^n\dd x}
        \end{align*}
        Note that
        \begin{align*}
            \phi_n'(x) = -nc_n(1 + \cos x)^{n-1}\sin x
        \end{align*}
        which is positive on $[-\pi, 0)$ and negative on $(0, \pi]$, and thus
        \begin{align*}
            \max_{x \in \mathbb{D}}\phi_n(x) = \phi_n(\delta)
        \end{align*}
        So,
        \begin{align*}
            \int_\mathbb{D}\phi_n(x) \dd x = \frac{\int_\mathbb{D} (1 + \cos x)^n \dd x}{\int_\mathbb{T} (1 + \cos x)^n \dd x} \leq \frac{2\pi (1 + \cos \delta)^n}{\int_\mathbb{E}(1 + \cos x)^n \dd x}
        \end{align*}
        where $\mathbb{E} = [-\frac{\delta}{2}, \frac{\delta}{2}]$.  Again, since $\phi_n$ is decreasing on $\left(0, \frac{\pi}{2}\right]$ and $\phi$ is an even function,
        \begin{align*}
            \min_{x\in\mathbb{E}}\phi_n(x) = \phi_n\qty(\frac{\delta}{2})
        \end{align*}
        Thus,
        \begin{align*}
            \int_\mathbb{D}\phi_n(x) \dd x \leq \frac{2\pi (1 + \cos \delta)^n}{\int_\mathbb{E}(1 + \cos x)^n \dd x} \leq \frac{2\pi}{\delta}\qty(\frac{1 + \cos \delta}{1 + \cos \frac{\delta}{2}})^n
        \end{align*}
        but
        \begin{align*}
            \frac{1 + \cos \delta}{1 + \cos \frac{\delta}{2}} < 1
        \end{align*}
        since $\cos$ is a decreasing function on $[0, \pi]$.  Thus,
        \begin{align*}
            \lim_{n\rightarrow \infty}\frac{2\pi}{\delta}\qty(\frac{1 + \cos \delta}{1 + \cos \frac{\delta}{2}})^n = 0
        \end{align*}
        and by the comparison test,
        \begin{align*}
            \lim_{n\rightarrow \infty}\int_\mathbb{D}\phi_n(x) \dd x = 0
        \end{align*}
    \item
        \emph{Prove that if the set $\mathcal{P}$ of trigonometric polynomials is dense in the space of periodic continuous functions on $\mathbb{T}$ with the uniform norm, then $\mathcal{P}$ is dense in the space of all continuous functions on $\mathbb{T}$ with the $L^2$-norm.}
    \item
        \emph{Is $\mathcal{P}$ dense in the space of all continuous functions on $[0, 2\pi]$ with the uniform norm?}
\end{enumerate}

\section*{Hunter and Nachtergaele 7.2}
\emph{Suppose that $f\ :\ \mathbb{T} \rightarrow \mathbb{C}$ is a continuous function, and $$S_N = \frac{1}{\sqrt{2\pi}}\sum_{n=-N}^N \hat{f_n}e^{inx}$$ is the $N$\textsuperscript{th} partial sum of its Fourier seriers.}
\begin{enumerate}[\bf (a)]
    \item
        \emph{Show that $S_N = D_N * f$, where $D_N$ is the \emph{Dirichlet kernel} $$D_N(x) = \frac{1}{2\pi}\frac{\sin\qty[(N+\frac{1}{2})x]}{\sin \qty(\frac{x}{2})}.$$}

        For ease, let $\omega = e^{ix}$.  Then note
        \begin{align*}
            \sum_{n=0}^N \omega^n = \frac{1 - \omega^{N+1}}{1 - \omega},\ \ \ \ \text{and}\ \ \ \ \sum_{n=-N}^{-1}\omega^n = \frac{\omega^{-N} - 1}{1 - \omega}
        \end{align*}
        Then
        \begin{align*}
            \frac{1}{2\pi}\sum_{n=-N}^N e^{inx} &= \frac{1}{2\pi}\sum_{n=-N}^N \omega^n = \frac{1}{2\pi}\frac{\omega^{-N} - \omega^{N+1}}{1 - \omega} = \frac{1}{2\pi}\frac{\omega^{-N - \frac{1}{2}} - \omega^{N + \frac{1}{2}}}{\omega^{-\frac{1}{2}} - \omega^{\frac{1}{2}}} \\
            &= \frac{1}{2\pi}\frac{\exp[ix[N + \frac{1}{2}]] - \exp[-ix[N+\frac{1}{2}]]}{\exp[ix[\frac{1}{2}]] - \exp[-ix[\frac{1}{2}]]} = \frac{1}{2\pi}\frac{\sin\qty[[N+\frac{1}{2}]x]}{\sin\qty[\frac{x}{2}]} = D_N(x)
        \end{align*}
        Then note
        \begin{align*}
            S_N &= \frac{1}{\sqrt{2\pi}}\sum_{n=-N}^N\hat{f}_n e^{inx} \\
            &= \frac{1}{\sqrt{2\pi}}\sum_{n=-N}^N\qty[\frac{1}{\sqrt{2\pi}}\int_{-\pi}^\pi f(y)e^{-iny}\dd y]e^{inx} \\
            &= \int_{-\pi}^\pi f(y)\qty(\frac{1}{2\pi}\sum_{n=-N}^N e^{in(x-y)})\dd y \\
            &= D_N * f
        \end{align*}
    \item
        \emph{Let $T_N$ be the mean of the first $N+1$ partial sums, $$T_N = \frac{1}{N+1}\qty(S_0 + S_1 + \dots + S_N) = \frac{1}{N+1}\sum_{i=0}^N S_i(x).$$  Show that $T_N = F_N * f$, where $F_N$ is the \emph{Fej\'{e}r kernel} $$F_N(x) = \frac{1}{2\pi(N + 1)}\qty(\frac{\sin\qty[(N + 1)\frac{x}{2}]}{\sin\qty(\frac{x}{2})})^2.$$}

        First note the following identity:
        \begin{align*}
            \frac{\sin^2\qty[(N+1)\frac{x}{2}]}{\sin\qty[\frac{x}{2}]} &= \frac{1 - \cos\qty[(N+1)x]}{2\sin\qty[\frac{x}{2}]}\ \ \ \ \text{by the power-reducing formulas} \\
            &= \frac{1}{2\sin\qty[\frac{x}{2}]}\Big(\qty[\cos(0x) - \cos(1x)] + \qty[\cos(1x) - \cos(2x)] + \dots \\
            &\qquad\dots+ \qty[\cos((N-1)x) - \cos(Nx)] + \qty[\cos(Nx) - \cos((N+1)x)]\Big) \\
            &\qquad\qquad\qquad\qquad\text{using a telescoping series} \\
            &= \frac{1}{2\sin\qty[\frac{x}{2}]}2\sin\qty[\frac{x}{2}]\sum_{i=0}^\infty\sin\qty[\frac{2i+1}{2}x] \\
            &= \sum_{i=0}^\infty\sin\qty[\frac{2i+1}{2}x]
        \end{align*}
        Then note that
        \begin{align*}
            F_N(x) &= \frac{1}{2\pi(N+1)}\qty(\frac{\sin\qty[(N+1)\frac{x}{2}]}{\sin\qty[\frac{x}{2}]})^2 \\
            &= \frac{1}{2\pi(N+1)\sin\qty[\frac{x}{2}]}\sum_{i=0}^\infty\sin\qty[\frac{2i+1}{2}x] \\
            &= \frac{1}{N+1}\sum_{i=0}^N\frac{1}{2\pi}\frac{\sin\qty[(i + \frac{1}{2})x]}{\sin\qty[\frac{x}{2}]} \\
            &= \frac{1}{N+1}\sum_{i=0}^N D_i(x)
        \end{align*}
        Lastly,
        \begin{align*}
            T_N(x) &= \frac{1}{N+1}\sum_{i=0}^N S_i(x) \\
            &= \frac{1}{N+1}\sum_{i=0}^N(D_i * f)(x)\ \ \ \ \text{by part {\bf (a)}} \\
            &= \frac{1}{N+1}\sum_{i=0}^N\int_\mathbb{T} f(y) D_i(x-y)\dd y \\
            &= \int_\mathbb{T} f(y) \qty[\frac{1}{N+1}\sum_{i=0}^N D_i(x-y)]\dd y \\
            &= \int_\mathbb{T} f(y) F_N(x-y)\dd y \\
            &= (F_N * f)(x)
        \end{align*}
    \item
        \emph{Which of the families $(D_N)$ and $(F_N)$ are approximate identities as $N\rightarrow \infty$?  What can you say about the uniform convergence of the partial sums $S_N$ and the averaged partial sums $T_N$ to $f$?} \\

        We know $(D_N)$ can not be an approximate identity since $$D_3(\pi) = \frac{1}{2\pi}\cdot\frac{\sin\qty[\frac{7}{2}\pi]}{\sin\qty[\frac{\pi}{2}]} = -\frac{1}{2\pi} < 0$$ and each function in an approximate identity must be nonnegative on $[-\pi, \pi]$.  We claim, however, that $(F_N)$ is an approximate identity.  First,
        \begin{align*}
            F_N(x) = \frac{1}{2\pi(N+1)}\qty(\frac{\sin\qty[(N+1)\frac{x}{2}]}{\sin\qty[\frac{x}{2}]})^2 \geq \frac{1}{2\pi(N+1)} > 0,\ \ \ \ \ \ \forall N \geq 0, \forall x \in \mathbb{T}
        \end{align*}
        Next we show
        \begin{align*}
            \int_\mathbb{T} F_N(x)\dd x = 1
        \end{align*}
        for all $N \geq 0$.
        \begin{align*}
            \int_\mathbb{T} F_N(x)\dd x &= \frac{1}{N+1}\int_\mathbb{T}\sum_{j=0}^N D_j(x)\dd x \\
            &= \frac{1}{N+1}\int_\mathbb{T}\sum_{j=0}^N \qty[\frac{1}{2\pi}\sum_{n=-j}^j e^{inx}] \dd x \\
            &= \frac{1}{2\pi(N+1)}\sum_{j=0}^N\sum_{n=-j}^j\int_\mathbb{T} e^{inx} \dd x\ \ \ \ \ \ \text{since the sums are finite} \\
            &= \frac{1}{2\pi(N+1)}\sum_{j=0}^N\qty[2\pi + \sum_{\substack{n=-j\\n\neq 0}}^j \qty[\dfrac{1}{in}\qty(\cos(nx) + i\sin(nx))]_{-\pi}^\pi] \\
            &= \frac{1}{2\pi(N+1)}\sum_{j=0}^N 2\pi \\
            &= \frac{2\pi(N+1)}{2\pi(N+1)} \\
            &= 1
        \end{align*}
        Lastly we show
        \begin{align*}
            \lim_{N\rightarrow \infty}\int_\mathbb{D}F_N(x) \dd x = 0
        \end{align*}
        where $\mathbb{D} = [-\pi, -\delta]\cup[\delta, \pi]$.  However,
        \begin{align*}
            \int_\mathbb{D}F_N(x) \dd x &= \frac{1}{2\pi(N+1)}\int_\mathbb{D}\qty(\frac{\sin\qty[(N+1)\frac{x}{2}]}{\sin\qty[\frac{x}{2}]})^2\dd x \\
            &\leq \frac{1}{2\pi(N+1)}\int_\mathbb{D} \qty(\frac{1}{\sin\qty[\frac{\delta}{2}]})^2 \dd x \\
            &= \frac{\pi - \delta}{\pi(N+1)\sin^2\qty[\frac{\delta}{2}]}
        \end{align*}
        since $\sin\qty[\frac{x}{2}]$ is a symmetric, increasing function on $[\delta, \pi]$.  But the sequence
        \begin{align*}
            \frac{\pi - \delta}{\pi(N+1)\sin^2\qty[\frac{\delta}{2}]} \rightarrow 0
        \end{align*}
        as $N \rightarrow \infty$.  Thus, by the comparison test,
        \begin{align*}
            \lim_{N\rightarrow\infty}\int_\mathbb{D}F_N(x)\dd x = 0
        \end{align*}
        This shows $(F_N)$ is an approximate identity.
\end{enumerate}

\section*{Hunter and Nachtergaele 7.3}
\emph{Prove that the sets $\{e_n\ |\ n \geq 1\}$ defined by $$e_n(x) = \sqrt{\frac{2}{\pi}}\sin nx,$$ and $\{f_n\ :\ n \geq 1\}$ defined by $$f_0(x) = \sqrt{\frac{1}{\pi}},\ \ \ \ \ f_n(x) = \sqrt{\frac{2}{\pi}}\cos nx\ \ \ \ \text{ for } n \geq 1,$$ are both orthonormal bases of $L^2([0,\pi])$.} \\

First we show $\{e_n\}_{n=1}^\infty$ and $\{f_n\}_{n=0}^\infty$ are orthonormal.  Suppose $n \neq m$.  Then
\begin{align*}
    \langle e_n, e_m \rangle &= \int_0^\pi e_n(x)e_m(x) \dd x \\
    &= \frac{2}{\pi}\int_0^\pi \sin(nx) \sin (mx) \dd x\\
    &= \frac{2}{\pi}\int_0^\pi \frac{1}{2}\qty[\cos(nx - mx) - \cos(nx + mx)]\dd x \\
    &= \frac{1}{\pi}\int_0^\pi\cos((n-m)x)\dd x - \frac{1}{\pi}\int_0^\pi\cos((n+m)x) \\
    &= \frac{1}{\pi}\qty[\frac{\sin((n-m)x)}{n-m} - \frac{\sin((n+m)x)}{n+m}]_0^\pi \\
    &= 0
    % &= \frac{1}{\pi}\qty[\frac{n\sin(n-m) + m\sin(n-m) - n\sin(n+m) + m\sin(n+m)}{n^2 - m^2}] \\
    % &= \frac{1}{\pi}\qty[\frac{(n+m)(\sin n \cos m - \cos n \sin m) - (n-m)(\sin n \cos m + \cos n \sin m)}{n^2 - m^2}] \\
    % &= \frac{1}{\pi}\qty[\frac{2m\sin n \cos m - 2n\cos n \sin m}{n^2 - m^2}]
\end{align*}
Also,
\begin{align*}
    \langle e_n, e_n \rangle &= \frac{2}{\pi}\int_0^\pi \sin^2(nx)\dd x \\
    &= \frac{1}{\pi} \int_0^\pi 1 - \cos\qty(2nx) \dd x \\
    &= \frac{1}{\pi} \qty[\pi - \frac{1}{2n}\sin(2n\pi)] \\
    &= \frac{1}{\pi}\pi\\ 
    &= 1
\end{align*}
Thus $\{e_n\}_{n=1}^\infty$ is orthonormal.  Let $n \geq 1$.
\begin{align*}
    \langle f_0, f_n\rangle &= \frac{\sqrt{2}}{\pi}\int_0^\pi\cos(nx)\dd x \\
    &= \frac{\sqrt{2}}{\pi}\frac{1}{n}\sin(nx)\Big|_0^\pi \\
    &= 0
\end{align*}
Let $1 \leq n < m$.  Then
\begin{align*}
    \langle f_n, f_m \rangle &= \frac{2}{\pi}\int_0^\pi \cos(nx)\cos(mx)\dd x \\
    &= \frac{1}{\pi}\int_0^\pi \qty[\cos((n-m)x) + \cos((n+m)x)]\dd x \\
    &= \frac{1}{\pi}\qty(\frac{\sin((n-m)x)}{n-m} + \frac{\sin((n+m)x)}{n+m})\Bigg|_0^\pi \\
    &= 0
\end{align*}
Also,
\begin{align*}
    \langle f_0, f_0 \rangle &= \frac{1}{\pi}\int_0^\pi\dd x = \frac{\pi}{\pi} = 1
\end{align*}
and for $n \geq 1$,
\begin{align*}
    \langle f_n, f_n \rangle &= \frac{2}{\pi}\int_0^\pi \cos^2(nx)\dd x \\
    &= \frac{1}{\pi}\int_0^\pi \qty(1 + \cos\qty(2nx)) \dd x \\
    &= \frac{1}{\pi}\qty[\pi + \qty(\frac{1}{2}{\sin(2nx)})_0^\pi] \\
    &= 1
\end{align*}
Thue $\{f_n\}_{n=0}^\infty$ is orthonormal.  Next we show $\{f_n\}_{n=0}^\infty$ and $\{e_n\}_{n=1}^\infty$ are each bases of $L^2[0, \pi]$.

Let $f \in L^2([0, \pi])$.  Then extend $f$ to its odd extension $f_{\text{odd}} \in L^2([-\pi, \pi])$ by
\begin{align*}
    f_{\text{odd}}(x) = \begin{cases}
        f(x) & \text{ if } x \in (0, \pi] \\
        0 & \text{ if } x = 0 \\
        -f(-x) & \text{ if } x \in [-\pi, 0)
    \end{cases}
\end{align*}
We know $\{e_n\}_{n=1}^\infty \cup \{f_n\}_{n=0}^\infty$ is an orthonormal basis of $L^2[-\pi, \pi]$ and thus $f_{\text{odd}}$ can be written as a Fourier series like so
\begin{align*}
    f_{\text{odd}}(x) = \frac{1}{2}f_0 + \sum_{n=1}^\infty\qty(a_nf_n + b_ne_n)
\end{align*}
But since $f_{\text{odd}}$ is constructed to be odd,
\begin{align*}
    f_{\text{odd}}(x) &= \sum_{n=1}^\infty b_ne_n
\end{align*}
Thus on $[0, \pi]$,
\begin{align*}
    f(x) = \sum_{n=1}^\infty e_n\sin(nx)
\end{align*}
Thus $\{e_n\}_{n=1}^\infty$ is a basis of $L^2[0, \pi]$.  Now extend $f$ to its even extension $f_{\text{even}} \in L^2[-\pi, \pi]$ be
\begin{align*}
    f_{\text{even}}(x) = \begin{cases}
        f(x) & \text{ if } x \in [0, \pi] \\
        f(-x) & \text{ if } x \in [-\pi, 0)
    \end{cases}
\end{align*}
Again, we know $\{e_n\}_{n=1}^\infty \cup \{f_n\}_{n=0}^\infty$ is an orthonormal basis of $L^2[-\pi, \pi]$ and thus $f_{\text{even}}$ can be written as a Fourier series like so
\begin{align*}
    f_{\text{even}}(x) = \frac{1}{2}f_0 + \sum_{n=1}^\infty\qty(a_nf_n + b_ne_n)
\end{align*}
But since $f_{\text{even}}$ is constructed to be even,
\begin{align*}
    f_{\text{even}}(x) = \frac{1}{2}f_0 + \sum_{n=1}^\infty a_nf_n
\end{align*}
Thus $\{f_n\}_{n=0}^\infty$ is a basis of $L^2[0, \pi]$.

\section*{Hunter and Nachtergaele 7.4}
\emph{Let $T, S \in L^2(\mathbb{T})$ be the triangular and square wave, respectively, defined by}
\begin{align*}
    T(x) = |x|,\ \ \ \ \text{ if } |x| \leq \pi,\ \ \ \ S(x) = \begin{cases}
        1 & \text{ if } 0 < x < \pi \\
        -1 & \text{ if } -\pi < x < 0
    \end{cases}
\end{align*}
\begin{enumerate}[\bf (a)]
    \item
        \emph{Compute the Fourier series of $T$ and $S$.} \\

        Since $T$ is an even function, we can represent $T$ with a cosine series
        \begin{align*}
            T(x) = \frac{1}{2}\hat{T}_0 + \sum_{n=1}^\infty \hat{T}_n\cos(n x)
        \end{align*}
        where
        \begin{align*}
            \hat{T}_0 &= \frac{1}{\pi}\int_\mathbb{T} T(x) \dd x\ \ \ \text{and}\\
            \hat{T}_n &= \frac{1}{\pi}\int_\mathbb{T} T(x) \cos (nx) \dd x,\ \ \ n = 1, 2, \dots
        \end{align*}
        Because $\cos$ is even and $T$ is even, $T \sin$ is even, and so
        \begin{align*}
            \hat{T}_0 = \frac{1}{\pi}\int_\mathbb{T} T(x) \dd x = \frac{2}{\pi}\int_0^\pi x \dd x = \frac{2}{\pi}\frac{\pi^2}{2} = \pi
        \end{align*}
        and for $n = 1, 2, \dots$,
        \begin{align*}
            \hat{T}_n &= \frac{2}{\pi}\int_0^\pi x\cos(nx) \dd x
        \end{align*}
        Utilizing integration by parts, we find
        \begin{align*}
            \hat{T}_n &= \frac{2}{\pi}\int_0^\pi x \cos (nx) \dd x \\
            &= \frac{2}{\pi}\qty[\qty(\frac{x}{n}\sin (nx))\Big|_0^\pi -\frac{1}{n}\int_0^\pi \sin(nx)\dd x] \\
            &= \frac{2}{\pi}\qty[\frac{1}{n^2}\cos (nx)]_0^\pi \\
            &= \frac{2}{\pi}\qty[\frac{(-1)^n - 1}{n^2}] \\
            &= \begin{cases}
                0 & \text{ if $n$ is even} \\
                -\dfrac{4}{\pi n^2} & \text{ if $n$ is odd}
            \end{cases}
        \end{align*}
        Thus,
        \begin{align*}
            \boxed{T(x) = \frac{\pi}{2} - \frac{4}{\pi}\sum_{n=1}^\infty \qty[\frac{1}{(2n-1)^2}\cos((2n-1)x)]}
        \end{align*}
        Since $S$ is an odd function, we can represent $S$ with a sin series
        \begin{align*}
            S(x) = \sum_{n=1}^\infty \hat{S}_n \sin(nx)
        \end{align*}
        where
        \begin{align*}
            \hat{S}_n = \frac{1}{\pi}\int_\mathbb{T} S(x)\sin(nx)\dd x,\ \ \ n = 1, 2, \dots
        \end{align*}
        Because $\sin$ is odd and $S$ is odd, $\sin S$ is even, and thus
        \begin{align*}
            \hat{S}_n &= \frac{1}{\pi}\int_\mathbb{T} S(x) \sin(nx) \dd x \\
            &= \frac{2}{\pi}\int_0^\pi \sin(nx)\dd x \\
            &= \frac{2}{\pi}\qty[-\frac{1}{n}\cos(nx)]_0^\pi \\
            &= -\frac{2}{\pi n}\qty((-1)^n - 1) \\
            &= \begin{cases}
                0 & \text{ if $n$ is even} \\
                \dfrac{4}{\pi n} & \text{ if $n$ is odd}
            \end{cases}
        \end{align*}
        Thus,
        \begin{align*}
            \boxed{S(x) = \frac{4}{\pi}\sum_{n=1}^\infty\qty[\frac{1}{(2n-1)}\sin((2n-1)x)]}
        \end{align*}
    \item
        \emph{Show that $T \in H^1(\mathbb{T})$ and $T' = S$.} \\

        First we turn $T(x)$ into a a Fourier series with $\{e^{inx}\}_{n \in \mathbb{Z}}$ as the basis using
        \begin{align*}
            \cos x = \frac{1}{2}\qty[e^{ix} + e^{-ix}]
        \end{align*}
        Thus,
        \begin{align*}
            T(x) &= \frac{\pi}{2} - \frac{4}{\pi}\sum_{n=1}^\infty \qty[\frac{1}{(2n-1)^2}\cos((2n-1)x)] \\
            &= \frac{\pi}{2} - \frac{2}{\pi}\sum_{n\in\mathbb{Z}} \frac{\exp[i(2n-1)x]}{(2n-1)^2} \\
            &= \frac{1}{\sqrt{2\pi}}\qty[\frac{\pi^2}{\sqrt{2\pi}} - \frac{4}{\sqrt{2\pi}}\sum_{n\in\mathbb{Z}}\frac{\exp[i(2n-1)x]}{(2n-1)^2}]
        \end{align*}
        To show $T \in H^1(\mathbb{T})$, we show
        \begin{align*}
            \sum_{n\in\mathbb{Z}} n^2 |\hat{T}_n|^2 < \infty
        \end{align*}
        but this is true because
        \begin{align*}
            \sum_{n\in\mathbb{Z}} n^2 |\hat{T}_n|^2 &= \frac{8}{\pi}\sum_{n\in\mathbb{Z}} \frac{(2n-1)^2}{(2n-1)^4} < \infty
        \end{align*}
        by the comparison test.  Thus $T \in H^1(\mathbb{T})$.

        Next note that $S(x)$ can be turned into a Fourier series with $\{e^{inx}\}_{n\in\mathbb{Z}}$ asa basis by using the following:
        \begin{align*}
            \sin x = \frac{1}{2i}\qty[e^{ix} - e^{-ix}]
        \end{align*}
        Thus,
        \begin{align*}
            S(x) &= \frac{4}{\pi}\sum_{n=1}^\infty\qty[\frac{1}{(2n-1)}\sin((2n-1)x)] \\
            &= -\frac{2i}{\pi}\sum_{n\in\mathbb{Z}}\frac{\exp[i(2n-1)x]}{2n-1}
        \end{align*}
        We can explicitly calculuate $in\hat{T}_n$ for each $n$:
        \begin{align*}
            T' = \frac{1}{\sqrt{2\pi}}\qty[\frac{\pi^2}{\sqrt{2\pi}}(0i) - \frac{4}{\sqrt{2\pi}}\sum_{n\in\mathbb{Z}}((2n-1)i)\frac{\exp[i(2n-1)x]}{(2n-1)^2}] &= -\frac{2i}{\pi}\sum_{n\in\mathbb{Z}}\frac{\exp[i(2n-1)x]}{2n-1} = S
        \end{align*}
    \item
        \emph{Show that $S \not\in H^1(\mathbb{T})$.} \\

        To show $S \not\in H^1(\mathbb{T})$, we show
        \begin{align*}
            \sum_{n\in\mathbb{Z}} n^2 |\hat{S}_n|^2 < \infty
        \end{align*}
        but this is true because
        \begin{align*}
            \sum_{n\in\mathbb{Z}} n^2 |\hat{S}_n|^2 &= \frac{4}{\pi^2}\sum_{n\in\mathbb{Z}} \frac{(2n-1)^2}{(2n-1)^2} = \infty
        \end{align*}
        by the $n$\textsuperscript{th} term test.  Thus $S \not\in H^1(\mathbb{T})$.
\end{enumerate}

\section*{Hunter and Nachtergaele 7.5}
\emph{Consider $f\ :\ \mathbb{T}^d \rightarrow \mathbb{C}$ defined by $$f(x) = \sum_{n\in\mathbb{Z}^d}a_n e^{i n \cdot x},$$ where $x = (x_1, x_2, \dots, x_d)$, $n = (n_1, n_2, \dots, n_d)$, and $n\cdot x = n_1x_1 + n_2x_2 + \dots + n_dx_d$.  Prove that if $$\sum_{n\in\mathbb{Z}^d}|n|^{2k}|a_n|^2 <\infty$$ for some $k > \frac{d}{2}$, then $f$ is continuous.} \\

Let $f \in H^k(\mathbb{T}^d)$ with $k > \frac{1}{2}$.  Define the partial sums $S_N$ of the Fourier series of $f$ by
\begin{align*}
    S_N(x) = \sum_{([-N, N]\cap \mathbb{Z})^d}\hat{f}_n e^{i n \cdot x}
\end{align*}
and define the norm of the $k$\textsuperscript{th} weak derivative of $f$ as
\begin{align*}
    \norm{f^{k}}^2 = \sum_{n \in \mathbb{Z}^d}|n|^{2k}|\hat{f}_n|^2
\end{align*}
We will show the sequence $S_N \rightarrow f$ uniformly by showing $(S_N)_N$ is a Cauchy sequence and since $C(\mathbb{T}^d)$ is complete with respect to the supremum norm, this implies the limit of $(S_N)_N$ is contained in $C(\mathbb{T}^d)$.
\begin{align*}
    \norm{S_N - S_M}_\infty &= \norm{\sum_{n \in ((\pm N, \pm M]\cap \mathbb{Z})^d} \hat{f}_n e^{i n \cdot x}}_\infty \\
    &\leq \sum_{n \in ((\pm N, \pm M]\cap \mathbb{Z})^d}|\hat{f}_n||e^{in\cdot x}| \\
    &\qquad\text{by the Triangle Inequality} \\
    &= \sum_{n \in ((\pm N, \pm M]\cap \mathbb{Z})^d}|\hat{f}_n| \\
    &= \sum_{n \in ((\pm N, \pm M]\cap \mathbb{Z})^d} |n|^k |\hat{f}_n|\frac{1}{|n|^k} \\
    &\leq \sqrt{\sum_{n \in ((\pm N, \pm M]\cap \mathbb{Z})^d}|n|^{2k}|\hat{f}_n^2|}\cdot\sqrt{\sum_{n \in ((\pm N, \pm M]\cap \mathbb{Z})^d}\frac{1}{|n|^{2k}}} \\
    &\qquad\text{by the Cauchy-Schwarz Inequality} \\
    &\leq \norm{f^{(k)}}\sqrt{\sum_{n \in ((\pm N, \pm M]\cap \mathbb{Z})^d}\frac{1}{|n|^{2k}}} \\
    &\qquad\text{since the Fourier transform is an isomorphism and thus preserves norm} \\
    &\leq \norm{f^{(k)}}_\infty\sqrt{|\mathbb{S}^{d-1}|\int_N^\infty \frac{r^{d-1}}{r^{2k}}\dd r} \\
    &\qquad\text{where $|\mathbb{S}^{d-1}|$ is the area of the unit sphere in $d$ dimensions} \\
    &= \norm{f^{(k)}}_\infty\sqrt{|\mathbb{S}^{d-1}|}\sqrt{\frac{r^{d-2k}}{d-2k}\Bigg|_{N}^{\infty}} \\
    &= \begin{cases}
        \infty & \text{ if } \frac{d}{2}\geq k \\
        \norm{f^{(k)}}_\infty\sqrt{|\mathbb{S}^{d-1}|}\qty((2k-d)N^{2k-d})^{-\frac{1}{2}} & \text{ if } \frac{d}{2} < k
    \end{cases}
\end{align*}
Supposing $\frac{d}{2} < k$,
\begin{align*}
    \norm{S_N - S_M}_\infty \leq \frac{\norm{f^{(k)}}_\infty\sqrt{|\mathbb{S}^{d-1}|}}{\sqrt{(2k-d)N^{2k-d}}}
\end{align*}
which goes to zero as $N \rightarrow \infty$.  Thus $(S_N)_N$ is a Cauchy sequence and thus converges to a limit in $C(\mathbb{T}^d)$.  But $S_N$ are the partial sums of the Fourier series of $f$, and thus $S_N \rightarrow f$.  Thus $f \in C(\mathbb{T}^d)$, i.e.~$f$ is continuous.

\section*{Hunter and Nachtergaele 7.6}
\emph{Suppose that $f \in H^1([a,b])$ and $f(a) = f(b) = 0$.  Prove the \emph{Poincar\'{e} inequality} $$\int_a^b|f(x)|^2 \dd x \leq \frac{(b-a)^2}{\pi^2}\int_a^b|f'(x)|^2 \dd x.$$}

\section*{Hunter and Nachtergaele 7.7}
\emph{Solve the following initial-boundary value problem for the heat equation,}
\begin{align*}
    \begin{array}{ll}
    u_t = u_{xx}, & \\
    u(0, t) = 0,\ \ \ u(L,t) = 0 & \text{ for } t > 0 \\
    u(x, 0) = f(x) & \text{ for } 0 \leq x \leq L
    \end{array}
\end{align*}

Suppose $u(x,t) = F(x)G(t)$ is a solution.  Then
\begin{align*}
    u_t &= u_{xx} \\
    \implies F(x)G'(t) &= F''(x)G(t) \\
    \implies \frac{F''(x)}{F(x)} &= \frac{G'(t)}{G(t)}
\end{align*}
Since the left hand side is a function of $x$ and the right hand side is a function of $t$, they can only be equal if they are both constant, i.e.
\begin{align*}
    \frac{F''(x)}{F(x)} = C = \frac{G'(t)}{G(t)}
\end{align*}
for some $C \in \mathbb{R}$.  Thus,
\begin{align}
    G'(t) - C G(t) &= 0,\ \ \ \text{and} \\
    F''(x) - C F(x) &= 0
\end{align}
The solutions of (1) are
\begin{align*}
    G(t) = c_1 e^{Ct}
\end{align*}
Let $\lambda = \sqrt{C}$.  If $C \neq 0$, the solutions of (2) are
\begin{align*}
    F(x) = c_1 e^{\lambda x} + c_2 e^{-\lambda x}
\end{align*}
The initial condition
\begin{align*}
    u(0,t) = 0 \implies F(0)G(t) = 0 \implies F(0) = 0
\end{align*}
provided $u$ is not the trivial solution.  Similarly,
\begin{align*}
    F(L) = 0
\end{align*}
If $C > 0$,
\begin{align*}
    F(0) = 0 \implies 0 = c_1 + c_2 \implies F(x) = c_1\qty(e^{\lambda x} - e^{-\lambda x})
\end{align*}
Also,
\begin{align*}
    F(L) = 0 \implies 0 = c_1\qty(e^{\lambda L} - e^{-\lambda L}) \implies c_1 = 0
\end{align*}
Thus $u$ is the trivial solution.  If $C = 0$, then either $F'' = 0$ or $F \equiv 0$, but regardless, if $F'' = 0$, the initial conditions imply that $F \equiv 0$.  So let $C < 0$ and define $\lambda = \sqrt{-C}$.  Then
\begin{align*}
    F(x) = c_1 \sin(\lambda x) + c_2 \cos(\lambda x)
\end{align*}
Then
\begin{align*}
    F(0) = 0 \implies 0 = c_2 \implies F(x) = c_1 \sin(\lambda x)
\end{align*}
Also,
\begin{align*}
    F(L) = 0 \implies 0 = c_1 \sin(\lambda L) \implies \lambda L = \pi n
\end{align*}
for integer values $n$.  Thus $\lambda = \frac{n\pi}{L}$ for $n = \pm 1, \pm 2, \dots$.  Note $n \neq 0$ since that would imply $\lambda^2 = 0 = C$.  Thus,
\begin{align*}
    u(t,x) = \sum_{n=1}^\infty c_n \exp\qty(-\frac{n^2 \pi^2}{L^2}t)\sin\qty(\frac{n\pi}{L}x)
\end{align*}
The initial condition $u(0,x) = f(x)$ implies
\begin{align*}
    f(x) = \sum_{n=1}^\infty c_n\sin\qty(\frac{n\pi}{L}x)
\end{align*}
This is a Fourier series, and thus the coefficients $c_n$ are given by
\begin{align*}
    c_n = \frac{2}{L}\int_0^L f(x) \sin\qty(\frac{n\pi}{L}x)\dd x
\end{align*}
Thus the full solution is
\begin{align*}
    u(t,x) = \sum_{n=1}^\infty \qty(\frac{2}{L}\int_0^L \qty[f(x) \sin\qty(\frac{n\pi}{L}x)]\dd x \cdot \exp\qty(-\frac{n^2 \pi^2}{L^2}t)\cdot\sin\qty(\frac{n\pi}{L}x))
\end{align*}

\end{document}
