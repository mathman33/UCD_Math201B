\documentclass[12pt]{article}

\usepackage{amssymb, amsmath, amsfonts}
\usepackage{moreverb}
\usepackage{graphicx}
\usepackage{enumerate}
\usepackage[margin=0.75in]{geometry}
\usepackage{graphics}
\usepackage{color}
\usepackage{array}
\usepackage{float}
\usepackage{hyperref}
\usepackage{textcomp}
\usepackage{bbold}
\usepackage{cancel}
\usepackage{alltt}
\usepackage{physics}
\usepackage{mathtools}
\usepackage{amsthm}
\usepackage{tikz}
\usetikzlibrary{positioning}
\usetikzlibrary{arrows}
\usepackage{pgfplots}
\usepackage{bigints}
\allowdisplaybreaks
\pgfplotsset{compat=1.12}

\newcommand{\E}{\varepsilon}

\theoremstyle{plain}
\newtheorem*{theorem*}{Theorem}
\newtheorem{theorem}{Theorem}
\newtheorem*{lemma*}{Lemma}
\newtheorem{lemma}{Lemma}

\newenvironment{definition}[1][Definition]{\begin{trivlist}
\item[\hskip \labelsep {\bfseries #1}]}{\end{trivlist}}

\title{\bf HW \#2}
\author{\bf Sam Fleischer}
\date{\bf January 29, 2015}

\pgfplotsset{compat=1.12}

\begin{document}
\textbf{MATH 201B \hfill Analysis \ \ \hfill Winter 2016\ \ \ }

{\let\newpage\relax\maketitle}

\section*{Hunter and NachterGaele 7.1}
\emph{Let $\phi_n$ be the functions defined in (7.7) $$\phi_n(x) = c_n(1 + \cos x)^n$$ where $c_n$ is chosen such that $$\int_\mathbb{T}\phi_n(x) \dd x = 1$$ for all $n$.}
\begin{enumerate}[\bf (a)]
    \item
        \emph{Prove (7.5).  $$\lim_{n\rightarrow \infty}\int_{\delta\leq|x|\leq \pi}\phi_n(x) \dd x = 0$$ for every $\delta > 0$.} \\

        Let $\delta > 0$ and for ease, define $\mathbb{D} = [-\pi, -\delta]\cup[\delta, \pi]$.  
        \begin{align*}
            \int_\mathbb{D} \phi_n(x)\dd x = \frac{\int_\mathbb{D} (1 + \cos x)^n \dd x}{\int_\mathbb{T} (1 + \cos x)^n\dd x}
        \end{align*}
        since
        \begin{align*}
            c_n = \frac{1}{\int_\mathbb{T}(1 + \cos x)^n\dd x}
        \end{align*}
        Note that
        \begin{align*}
            \phi_n'(x) = -nc_n(1 + \cos x)^{n-1}\sin x
        \end{align*}
        which is positive on $[-\pi, 0)$ and negative on $(0, \pi]$, and thus
        \begin{align*}
            \max_{x \in \mathbb{D}}\phi_n(x) = \phi_n(\delta)
        \end{align*}
        So,
        \begin{align*}
            \int_\mathbb{D}\phi_n(x) \dd x = \frac{\int_\mathbb{D} (1 + \cos x)^n \dd x}{\int_\mathbb{T} (1 + \cos x)^n \dd x} \leq \frac{2\pi (1 + \cos \delta)^n}{\int_\mathbb{E}(1 + \cos x)^n \dd x}
        \end{align*}
        where $\mathbb{E} = [-\frac{\delta}{2}, \frac{\delta}{2}]$.  Again, since $\phi_n$ is decreasing on $\left(0, \frac{\pi}{2}\right]$ and $\phi$ is an even function,
        \begin{align*}
            \min_{x\in\mathbb{E}}\phi_n(x) = \phi_n\qty(\frac{\delta}{2})
        \end{align*}
        Thus,
        \begin{align*}
            \int_\mathbb{D}\phi_n(x) \dd x \leq \frac{2\pi (1 + \cos \delta)^n}{\int_\mathbb{E}(1 + \cos x)^n \dd x} \leq \frac{2\pi}{\delta}\qty(\frac{1 + \cos \delta}{1 + \cos \frac{\delta}{2}})^n
        \end{align*}
        but
        \begin{align*}
            \frac{1 + \cos \delta}{1 + \cos \frac{\delta}{2}} < 1
        \end{align*}
        since $\cos$ is a decreasing function on $[0, \pi]$.  Thus,
        \begin{align*}
            \lim_{n\rightarrow \infty}\frac{2\pi}{\delta}\qty(\frac{1 + \cos \delta}{1 + \cos \frac{\delta}{2}})^n = 0
        \end{align*}
        and by the comparison test,
        \begin{align*}
            \lim_{n\rightarrow \infty}\int_\mathbb{D}\phi_n(x) \dd x = 0
        \end{align*}
    \item
        \emph{Prove that if the set $\mathcal{P}$ of trigonometric polynomials is dense in the space of periodic continuous functions on $\mathbb{T}$ with the uniform norm, then $\mathcal{P}$ is dense in the space of all continuous functions on $\mathbb{T}$ with the $L^2$-norm.}
    \item
        \emph{Is $\mathcal{P}$ dense in the space of all continuous functions on $[0, 2\pi]$ with the uniform norm?}
\end{enumerate}

\section*{Hunter and NachterGaele 7.2}
\emph{Suppose that $f\ :\ \mathbb{T} \rightarrow \mathbb{C}$ is a continuous function, and $$S_N = \frac{1}{\sqrt{2\pi}}\sum_{n=-N}^N \hat{f_n}e^{inx}$$ is the $N$\textsuperscript{th} partial sum of its Fourier seriers.}
\begin{enumerate}[\bf (a)]
    \item
        \emph{Show that $S_N = D_N * f$, where $D_N$ is the \emph{Dirichlet kernel} $$D_N(x) = \frac{1}{2\pi}\frac{\sin\qty[(N+\frac{1}{2})x]}{\sin \qty(\frac{x}{2})}.$$}
    \item
        \emph{Let $T_N$ be the mean of the first $N+1$ partial sums, $$T_N = \frac{1}{N+1}.$$  Show that $T_N = F_N * f$, where $F_N$ is the \emph{Fej\'{e}r kernel} $$F_N(x) = \frac{1}{2\pi(N + 1)}\qty(\frac{\sin\qty[(N + 1)\frac{x}{2}]}{\sin\qty(\frac{x}{2})})^2.$$}
    \item
        \emph{Which of the families $(D_N)$ and $(F_N)$ are approximate identities as $N\rightarrow \infty$?  What can you say about the uniform convergence of the partial sums $S_N$ and the averaged partial sums $T_N$ to $f$?}
\end{enumerate}

\section*{Hunter and NachterGaele 7.3}
\emph{Prove that the sets $\{e_n\ |\ n \geq 1\}$ defined by $$e_n(x) = \sqrt{\frac{2}{\pi}}\sin nx,$$ and $\{f_n\ :\ n \geq 1\}$ defined by $$f_0(x) = \sqrt{\frac{1}{\pi}},\ \ \ \ \ f_n(x) = \sqrt{\frac{2}{\pi}}\cos nx\ \ \ \ \text{ for } n \geq 1,$$ are both orthonormal bases of $L^2([0,\pi])$.} \\

First we show $\{e_n\}_{n=1}^\infty$ and $\{f_n\}_{n=0}^\infty$ are orthonormal.  Suppose $n \neq m$.  Then
\begin{align*}
    \langle e_n, e_m \rangle &= \int_0^\pi e_n(x)e_m(x) \dd x \\
    &= \frac{2}{\pi}\int_0^\pi \sin(nx) \sin (mx) \dd x\\
    &= \frac{2}{\pi}\int_0^\pi \frac{1}{2}\qty[\cos(nx - mx) - \cos(nx + mx)]\dd x \\
    &= \frac{1}{\pi}\int_0^\pi\cos(x(n-m))\dd x - \frac{1}{\pi}\int_0^\pi\cos(x(n+m)) \\
    &= \frac{1}{\pi}\qty[\frac{\sin(x(n-m))}{n-m} - \frac{\sin(x(n+m))}{n+m}]_0^\pi \\
    &= 0
    % &= \frac{1}{\pi}\qty[\frac{n\sin(n-m) + m\sin(n-m) - n\sin(n+m) + m\sin(n+m)}{n^2 - m^2}] \\
    % &= \frac{1}{\pi}\qty[\frac{(n+m)(\sin n \cos m - \cos n \sin m) - (n-m)(\sin n \cos m + \cos n \sin m)}{n^2 - m^2}] \\
    % &= \frac{1}{\pi}\qty[\frac{2m\sin n \cos m - 2n\cos n \sin m}{n^2 - m^2}]
\end{align*}
Also,
\begin{align*}
    \langle e_n, e_n \rangle &= \frac{2}{\pi}\int_0^\pi \sin^2(nx)\dd x \\
    &= \frac{1}{\pi} \int_0^\pi 1 - \cos\qty(2nx) \dd x \\
    &= \frac{1}{\pi} \qty[\pi - \frac{1}{2n}\sin(2n\pi)] \\
    &= \frac{1}{\pi}\pi\\ 
    &= 1
\end{align*}
Thus $\{e_n\}_{n=1}^\infty$ is orthonormal.

Let $f \in L^2([0, 1])$.  Then extend $f$ to its odd expansion $\tilde{f} \in L^2([-1, 1])$ by
\begin{align*}
    \tilde{f}(x) = \begin{cases}
        f(x) & \text{ if } x \in (0, 1) \\
        0 & \text{ if } x = 0 \\
        -f(x) & \text{ if } x \in (-1, 0)
    \end{cases}
\end{align*}
Then $\{e_n\}_{n=1}^\infty \cup \{f_n\}_{n=0}^\infty$.

\section*{Hunter and NachterGaele 7.4}
\emph{Let $T, S \in L^2(\mathbb{T})$ be the triangular and square wave, respectively, defined by}
\begin{align*}
    T(x) = |x|,\ \ \ \ \text{ if } |x| \leq \pi,\ \ \ \ S(x) = \begin{cases}
        1 & \text{ if } 0 < x < \pi \\
        -1 & \text{ if } -\pi < x < 0
    \end{cases}
\end{align*}
\begin{enumerate}[\bf (a)]
    \item
        \emph{Compute the Fourier series of $T$ and $S$.} \\

        Since $T$ is an even function, we can represent $T$ with a cosine series
        \begin{align*}
            T(x) = \frac{1}{2}\hat{T}_0 + \sum_{n=1}^\infty \hat{T}_n\cos(n x)
        \end{align*}
        where
        \begin{align*}
            \hat{T}_0 &= \frac{1}{\pi}\int_\mathbb{T} T(x) \dd x\ \ \ \text{and}\\
            \hat{T}_n &= \frac{1}{\pi}\int_\mathbb{T} T(x) \cos (nx) \dd x,\ \ \ n = 1, 2, \dots
        \end{align*}
        Because $\cos$ is even and $T$ is even, $T \sin$ is even, and so
        \begin{align*}
            \hat{T}_0 = \frac{1}{\pi}\int_\mathbb{T} T(x) \dd x = \frac{2}{\pi}\int_0^\pi x \dd x = \frac{2}{\pi}\frac{\pi^2}{2} = \pi
        \end{align*}
        and for $n = 1, 2, \dots$,
        \begin{align*}
            \hat{T}_n &= \frac{2}{\pi}\int_0^\pi x\cos(nx) \dd x
        \end{align*}
        Utilizing integration by parts, we find
        \begin{align*}
            \hat{T}_n &= \frac{2}{\pi}\int_0^\pi x \cos (nx) \dd x \\
            &= \frac{2}{\pi}\qty[\qty(\frac{x}{n}\sin (nx))\Big|_0^\pi -\frac{1}{n}\int_0^\pi \sin(nx)\dd x] \\
            &= \frac{2}{\pi}\qty[\frac{1}{n^2}\cos (nx)]_0^\pi \\
            &= \frac{2}{\pi}\qty[\frac{(-1)^n - 1}{n^2}] \\
            &= \begin{cases}
                0 & \text{ if $n$ is even} \\
                -\dfrac{4}{\pi n^2} & \text{ if $n$ is odd}
            \end{cases}
        \end{align*}
        Thus,
        \begin{align*}
            \boxed{T(x) = \frac{\pi}{2} - \frac{4}{\pi}\sum_{n=1}^\infty \qty[\frac{1}{(2n-1)^2}\cos((2n-1)x)]}
        \end{align*}
        Since $S$ is an odd function, we can represent $S$ with a sin series
        \begin{align*}
            S(x) = \sum_{n=1}^\infty \hat{S}_n \sin(nx)
        \end{align*}
        where
        \begin{align*}
            \hat{S}_n = \frac{1}{\pi}\int_\mathbb{T} S(x)\sin(nx)\dd x,\ \ \ n = 1, 2, \dots
        \end{align*}
        Because $\sin$ is odd and $S$ is odd, $\sin S$ is even, and thus
        \begin{align*}
            \hat{S}_n &= \frac{1}{\pi}\int_\mathbb{T} S(x) \sin(nx) \dd x \\
            &= \frac{2}{\pi}\int_0^\pi \sin(nx)\dd x \\
            &= \frac{2}{\pi}\qty[-\frac{1}{n}\cos(nx)]_0^\pi \\
            &= -\frac{2}{\pi n}\qty((-1)^n - 1) \\
            &= \begin{cases}
                0 & \text{ if $n$ is even} \\
                \dfrac{4}{\pi n} & \text{ if $n$ is odd}
            \end{cases}
        \end{align*}
        Thus,
        \begin{align*}
            \boxed{S(x) = \frac{4}{\pi}\sum_{n=1}^\infty\qty[\frac{1}{(2n-1)}\sin((2n-1)x)]}
        \end{align*}
    \item
        \emph{Show that $T \in H^1(\mathbb{T})$ and $T' = S$.}
    \item
        \emph{Show that $S \not\in H^1(\mathbb{T})$.}
\end{enumerate}

\section*{Hunter and NachterGaele 7.5}
\emph{Consider $f\ :\ \mathbb{T}^d \rightarrow \mathbb{C}$ defined by $$f(x) = \sum_{n\in\mathbb{Z}^d}a_n e^{i n \cdot x},$$ where $x = (x_1, x_2, \dots, x_d)$, $n = (n_1, n_2, \dots, n_d)$, and $n\cdot x = n_1x_1 + n_2x_2 + \dots + n_dx_d$.  Prove that if $$\sum_{n\in\mathbb{Z}^d}|n|^{2k}|a_n|^2 <\infty$$ for some $k > \frac{d}{2}$, then $f$ is continuous.}

\section*{Hunter and NachterGaele 7.6}
\emph{Suppose that $f \in H^1([a,b])$ and $f(a) = f(b) = 0$.  Prove the \emph{Poincar\'{e} inequality} $$\int_a^b|f(x)|^2 \dd x \leq \frac{(b-a)^2}{\pi^2}\int_a^b|f'(x)|^2 \dd x.$$}

\section*{Hunter and NachterGaele 7.7}
\emph{Solve the following initial-boundary value problem for the heat equation,}
\begin{align*}
    \begin{array}{ll}
    u_t = u_{xx}, & \\
    u(0, t) = 0,\ \ \ u(L,t) = 0 & \text{ for } t > 0 \\
    u(x, 0) = f(x) & \text{ for } 0 \leq x \leq L
    \end{array}
\end{align*}

Suppose $u(x,t) = F(x)G(t)$ is a solution.  Then
\begin{align*}
    u_t &= u_{xx} \\
    \implies F(x)G'(t) &= F''(x)G(t) \\
    \implies \frac{F''(x)}{F(x)} &= \frac{G'(t)}{G(t)}
\end{align*}
Since the left hand side is a function of $x$ and the right hand side is a function of $t$, they can only be equal if they are both constant, i.e.
\begin{align*}
    \frac{F''(x)}{F(x)} = C = \frac{G'(t)}{G(t)}
\end{align*}
for some $C \in \mathbb{R}$.  Thus,
\begin{align}
    F''(x) - C F(x) &= 0,\ \ \ \text{and} \\
    G'(t) - C G(t) &= 0
\end{align}
Let $\lambda = \sqrt{|C|}$.  The solutions of (1) are
\begin{align*}
    F(x) = c_1 e^{\lambda it} + c_2 e^{-\lambda it}
\end{align*}

\end{document}
