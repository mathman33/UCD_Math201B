\documentclass{article} % A4 paper and 11pt font size
\setcounter{secnumdepth}{0}

\usepackage{amssymb, amsmath, amsfonts}
\usepackage{moreverb}
\usepackage{graphicx}
\usepackage{enumerate}
\usepackage{graphics}
\usepackage[margin=1.25in]{geometry}
\usepackage{color}
\usepackage{tocloft}
\renewcommand{\cftsecleader}{\cftdotfill{\cftdotsep}}
\usepackage{array}
\usepackage{float}
\usepackage{hyperref}
\usepackage{textcomp}
\usepackage[makeroom]{cancel}
\usepackage{bbold}
\usepackage{alltt}
\usepackage{physics}
\usepackage{mathtools}
\usepackage[normalem]{ulem}
\usepackage{amsthm}
\usepackage{tikz}
\usetikzlibrary{positioning}
\usetikzlibrary{arrows}
\usepackage{pgfplots}
\usepackage{bigints}
\allowdisplaybreaks
\pgfplotsset{compat=1.12}

\theoremstyle{plain}
\newtheorem*{theorem*}{Theorem}
\newtheorem{theorem}{Theorem}
\newtheorem*{lemma*}{Lemma}
\newtheorem{lemma}{Lemma}

\newenvironment{definition}[1][Definition]{\begin{trivlist}
\item[\hskip \labelsep {\bfseries #1}]}{\end{trivlist}}

\newcommand{\E}{\varepsilon}
\def\Rl{\mathbb{R}}
\def\Cx{\mathbb{C}}

\usepackage[T1]{fontenc} % Use 8-bit encoding that has 256 glyphs
\usepackage{fourier} % Use the Adobe Utopia font for the document - comment this line to return to the LaTeX default
\usepackage[english]{babel} % English language/hyphenation

\usepackage{sectsty} % Allows customizing section commands
\allsectionsfont{\centering \normalfont\scshape} % Make all sections centered, the default font and small caps

\usepackage{fancyhdr} % Custom headers and footers
\pagestyle{fancy} % Makes all pages in the document conform to the custom headers and footers
\fancyhead[L]{\bf Sam Fleischer}
\fancyhead[C]{\bf UC Davis \\ Analysis (MAT201B)} % No page header - if you want one, create it in the same way as the footers below
\fancyhead[R]{\bf Winter 2016}

\fancyfoot[L]{\bf } % Empty left footer
\fancyfoot[C]{\bf \thepage} % Empty center footer
\fancyfoot[R]{\bf } % Page numbering for right footer
\renewcommand{\headrulewidth}{0pt} % Remove header underlines
\renewcommand{\footrulewidth}{0pt} % Remove footer underlines
\setlength{\headheight}{25pt} % Customize the height of the header

\newcommand{\VEC}[2]{\left\langle #1, #2 \right\rangle}
\newcommand{\ran}{\text{\rm ran }}
\newcommand{\Hilb}{\mathcal{H}}

\newcommand{\problem}[1]{
\vspace{.375cm}
\begin{minipage}{\textwidth}
    \begin{center}
        \noindent\rule{5cm}{1pt}
    \end{center}
    \section{\bf #1}
    \begin{center}
        \noindent\rule{5cm}{1pt}
    \end{center}
    \vspace{0.25cm}
\end{minipage}
}

\numberwithin{equation}{section} % Number equations within sections (i.e. 1.1, 1.2, 2.1, 2.2 instead of 1, 2, 3, 4)
\numberwithin{figure}{section} % Number figures within sections (i.e. 1.1, 1.2, 2.1, 2.2 instead of 1, 2, 3, 4)
\numberwithin{table}{section} % Number tables within sections (i.e. 1.1, 1.2, 2.1, 2.2 instead of 1, 2, 3, 4)

\setlength\parindent{0pt} % Removes all indentation from paragraphs - comment this line for an assignment with lots of text

\newcommand{\horrule}[1]{\rule{\linewidth}{#1}} % Create horizontal rule command with 1 argument of height

\title{ 
\normalfont \normalsize 
\textsc{UC Davis, Analysis (MAT201B), Winter 2016} \\ [25pt] % Your university, school and/or department name(s)
\horrule{2pt} \\[0.4cm] % Thin top horizontal rule
\Huge Homework \#7 \\ % The assignment title
\horrule{2pt} \\[0.5cm] % Thick bottom horizontal rule
}

\author{\huge Sam Fleischer} % Your name

\date{March 14, 2016} % Today's date or a custom date

\begin{document}\thispagestyle{empty}

\maketitle % Print the title

\makeatletter
\@starttoc{toc}
\makeatother

\pagebreak

%%%%%%%%%%%%%%%%%%%%%%%%%%%%%%%%%%%%%%
\problem{Hunter and Nachtergaele 9.1}
\emph{Prove that $\rho(A^*) = \overline{\rho(A)}$, where $\overline{\rho(A)}$ is the set $\{\lambda \in \Cx\ |\ \overline{\lambda} \in \rho(A)\}$.}
\begin{proof}
    First note
    \begin{align*}
        (A^* - \lambda I) = (A^* - (\overline{\lambda}I)^*) = (A - \overline{\lambda}I)^*,
    \end{align*}
    and since $(A - \overline{\lambda}I) \in \mathcal{B}(\Hilb)$, then $(A - \overline{\lambda}I)$ is invertible if and only if $(A - \overline{\lambda}I)^*$ is invertible.  Thus
    \begin{align*}
        \lambda \in \rho(A^*) &\iff (A^* - \lambda I) \text{ invertible} \\
        &\iff (A - \overline{\lambda}I)^* \text{ invertible} \\
        &\iff (A - \overline{\lambda} I) \text{ invertible} \\
        &\iff \overline{\lambda} \in \rho(A) \\
        &\iff \lambda \in \overline{\rho(A)}
    \end{align*}
    Thus, $\rho(A^*) = \overline{\rho(A)}$.
\end{proof}









%%%%%%%%%%%%%%%%%%%%%%%%%%%%%%%%%%%%%%
\problem{Hunter and Nachtergaele 9.3}
\emph{Suppose that $A$ is a bounded linear operator on a Hilbert space and $\lambda,\mu \in \rho(A)$.  Prove that the resolvent $R_\lambda$ of $A$ satisfies the \emph{resolvent equation} $$R_\lambda - R_\mu = (\mu - \lambda)R_\lambda R_\mu.$$}
\begin{proof}
    If the resolvent $R_\lambda$ is defined as $R_\lambda = (\lambda I - A)^{-1}$, then
    \begin{align*}
        \qty(\mu I - A) - \qty(\lambda I - A) &= (\mu - \lambda)I \\
        \implies \qty(\lambda I - A)^{-1}\qty[\qty(\mu I - A) - \qty(\lambda I - A)]\qty(\mu I - A)^{-1} &= \qty(\lambda I - A)^{-1}\qty[(\mu - \lambda)I]\qty(\mu I - A)^{-1} \\
        \implies \qty(\lambda I - A)^{-1} - \qty(\mu I - A)^{-1} &= \qty(\mu - \lambda)(\lambda I - A)^{-1}\qty(\mu I - A)^{-1} \\
        \implies R_\lambda - R_\mu &= (\mu - \lambda)R_\lambda R_\mu
    \end{align*}
\end{proof}









%%%%%%%%%%%%%%%%%%%%%%%%%%%%%%%%%%%%%%
\problem{Hunter and Nachtergaele 9.4}
\emph{Prove that the spectrum of an orthogonal projection $P$ is either $\{0\}$, in which case $P = 0$, or $\{1\}$, in which case $P = I$, or else $\{0,1\}$.}
\begin{proof}
    Let $\lambda$ be an eigenvalue.  Then $Px = \lambda x$ for some nonzero vector $x$.  Clearly if $P \equiv 0$, then $0 = Px = \lambda x$ for some $x \neq 0$, which implies $\lambda = 0$.  Also, if $P \equiv I$, then $x = Px = \lambda x \implies (1 - \lambda)x = 0$ for some $x \neq 0$.  Thus $\lambda = 1$.  In general, suppose $P \not\equiv 0$ and $P \not\equiv 1$, then for $x \in \ran P$, $x = Px = \lambda x \implies \lambda = 1$.  For $x \not\in \ran P$, then since $P$ is an orthogonal projection, $x = y + z$ for some $y \in \ran P$ (i.e.~$Py = y$) and $z \in \ker P$ (i.e.~$Pz = 0$).  Thus $Py + Pz = y = \lambda x$.  Since $x \not\in \ran P$, $\lambda x \in \ran P$ only if $\lambda = 0$.  Thus the only eigenvalues of $P$ are $0$ and $1$ (i.e.~the point spectrum of $P$ is contained in $\{0,1\}$).

    Since orthogonal projections are bounded and self adjoint, then the residual specturm of $P$ is empty.

    Let $a \in \ran P$.  Then $(1 - \lambda)a \in \ran (P - \lambda I)$ (since $(P - \lambda I)a = Pa - \lambda a = (1 - \lambda)a$).  If $\lambda \neq 1$ then $a \in \ran (P - \lambda I)$ since $\ran(P - \lambda I)$ is closed under scalar multiplication.  Let $b \in \ker P$.  Then $-\lambda b \in \ran (P - \lambda I)$ (since $(P - \lambda I)b = Pb - \lambda b = -\lambda b$).  If $\lambda \neq 0$, then $b \in \ran (P - \lambda I)$ since $\ran(P - \lambda I)$ is closed under scalar multiplication.  Thus for $\lambda \in \Cx\setminus \{0,1\}$,
    \begin{align*}
        \ran P \cup \ker P \subset \ran (P - \lambda I)
    \end{align*}
    Since $P$ is an orthogonal projection,
    \begin{align*}
        \Hilb \subset \ran (P - \lambda I) \subset \Hilb
    \end{align*}
    and thus $\ran (P - \lambda I)$ is closed.  Thus the continuous specturm of $P$ is empty.

    Thus, $\sigma(P) = \{0,1\}$.
\end{proof}









%%%%%%%%%%%%%%%%%%%%%%%%%%%%%%%%%%%%%%
\problem{Hunter and Nachtergaele 9.5}
\emph{Let $A$ be a bounded, nonnegative operator on a complex Hilbert space.  Prove that $\sigma(A) \subset [0, \infty)$.}
\begin{proof}
    Since $A$ is nonnegative, then $\VEC{x}{Ax} \geq 0$ for all $x \in \Hilb$ and $A = A^*$.  Since $A$ is self-adjoint, its eigenvalues are real and $\sigma(A) \subset \qty[-\norm{A}, \norm{A}]$.  Let $\lambda$ be an eigenvalue.  Then for some $x \neq 0$,
    \begin{align*}
        0 \leq \VEC{x}{Ax} = \VEC{x}{\lambda x} = \lambda \VEC{x}{x} = \lambda \norm{x}^2 \\
        \implies 0 \leq \lambda
    \end{align*}
    Thus all eigenvalues are positive (i.e.~the point spectrum is contained in $[0, \infty)$).  Let $\lambda \in $ continuous spectrum.  Then $\ran(A - \lambda I)$ is dense $\in \Hilb$.  Then there is a sequence $(y_n) \in \ran(A - \lambda I)$ such that $y_n \rightarrow 0$.  Then there is a sequence $(x_n) \in \Hilb$ such that $(A - \lambda I)x_n = y_n$.  Then
    \begin{align*}
        \lim_{n\rightarrow \infty}\VEC{x_n}{(A - \lambda I)x_n} = \lim_{n\rightarrow \infty}\VEC{x_n}{y_n} = \VEC{\lim_{n\rightarrow\infty}x_n}{\lim_{n\rightarrow\infty}y_n} = \VEC{\lim_{n\rightarrow\infty}}{0} = 0.
    \end{align*}
    Also,
    \begin{align*}
        \VEC{x_n}{(A - \lambda I)x_n} = \VEC{x_n}{Ax_n} - \lambda \norm{x_n}^2 = b_n - \lambda \norm{x_n}^2.
    \end{align*}
    where $b_n = \VEC{x_n}{Ax_n} \geq 0$ for all $n$.  But
    \begin{align*}
        b_n - \lambda \norm{x_n}^2 \rightarrow 0
    \end{align*}
    and $\norm{x_n} \geq 0$ for all $n$, which implies $\lambda \geq 0$.  Thus the continuous spectrum is contained in $[0, \infty)$.  Thus,
    \begin{align*}
        \sigma(A) \subset [0, \infty).
    \end{align*}
\end{proof}









%%%%%%%%%%%%%%%%%%%%%%%%%%%%%%%%%%%%%%
\problem{Hunter and Nachtergaele 9.6}
\emph{Let $G$ be a multiplication operator on $L^2(\Rl)$ defined by $$Gf(x) = g(x)f(x),$$ where $g$ is continuous and bounded.  Prove that $G$ is a bounded linear operator on $L^2(\Rl)$ and that its spectrum is given by $$\sigma(G) = \overline{\{g(x)\ :\ x \in \Rl\}}.$$  Can an operator of this form have eigenvalues?}
\begin{proof}
    Let $\lambda \not\in \overline{\{g(x)\ :\ x \in \Rl\}}$.  Then $\exists \E$ such that $|g(x) - \lambda| > \E$ for all $x \in \Rl$.  Thus $\frac{1}{g(x) - \lambda}$ is well-defined and we can define the inverse of $(G - \lambda I)$ as
    \begin{align*}
        \qty(\qty(G - \lambda I)^{-1}f)(x) = \frac{1}{g(x) - \lambda}f(x)
    \end{align*}
    because $\forall f \in L^2(\Rl)$,
    \begin{align*}
        \qty(\qty(G - \lambda I)\qty(G - \lambda I)^{-1})f = \qty(\qty(G - \lambda I)^{-1}\qty(G - \lambda I))f = f
    \end{align*}
    Thus $\lambda \in \rho(G)$, i.e.~$\lambda \not\in\sigma(G)$, which shows
    \begin{align*}
        \sigma(G) \subset \overline{\{g(x)\ :\ x \in \Rl\}}
    \end{align*}

    Next, consider $\lambda \in \{g(x)\ :\ x \in \Rl\}$.  Then $\exists x_0 \in \Rl$ such that $g(x_0) = \lambda$.  Then consider the characteristic function on $\{x\ :\ |x - x_0| < 1\}$,
    \begin{align*}
        \mathcal{X}(x) = \begin{cases}
            1 & \text{ if } |x - x_0| < 1 \\
            0 & \text{ otherwise}
        \end{cases}
    \end{align*}
    Then $\mathcal{X} \not\in\ran(G - \lambda I)$ since the only candidate function $\mathcal{C}$ to map to $\mathcal{X}$ is
    \begin{align*}
        \mathcal{C}(x) = \frac{\mathcal{X}(x)}{g(x) - \lambda}
    \end{align*}
    but this function is not square-integrable, i.e.~$\mathcal{C} \not\in L^2(\Rl)$.  Thus $(G - \lambda I)$ is not surjective, which shows $\lambda \in \sigma(G)$.  Thus
    \begin{align*}
        \{g(x)\ :\ x\in\Rl\} \subset \sigma(G)
    \end{align*}
    However, since $\sigma(G)$ is closed (all spectrums are closed), then any closure of a subset of $\sigma(G)$ is also a subset of $\sigma(G)$.  Thus
    \begin{align*}
        \overline{\{g(x)\ :\ x\in\Rl\}} \subset \sigma(G)
    \end{align*}
    which shows
    \begin{align*}
        \overline{\{g(x)\ :\ x\in\Rl\}} = \sigma(G)
    \end{align*}

    It is possible for $G$ to have eigenvalues.  Consider a function $g$ and $\lambda \in \Rl$ such that $\mu\qty(\left\{x\ :\ g(x) = \lambda\right\}) > 0$.  Then $\lambda$ is an eigenvalue of $G$ and any function $f$ such that $\text{supp} f \subset \left\{x\ :\ g(x) = \lambda\right\}$ is an eigenvector with respect to $\lambda$ since
    \begin{align*}
        (Gf)(x) = g(x)f(x) = \mathcal{X}_{\text{supp} f}g(x) f(x) + \cancelto{0}{\mathcal{X}_{\Rl\setminus\text{supp}f}g(x) f(x)} = \lambda f(x)
    \end{align*}
\end{proof}









%%%%%%%%%%%%%%%%%%%%%%%%%%%%%%%%%%%%%%
\problem{Hunter and Nachtergaele 9.7}
\emph{Let $K\ :\ L^2([0,1]) \rightarrow L^2([0,1])$ be the integral operator defined by $$Kf(x) = \int_0^x f(y) \dd y.$$}
\begin{enumerate}[\it a)]
    \item
        \emph{Find the adjoint operator $K^*$.}
        \begin{proof}
            \begin{align*}
                \VEC{f}{Kg} &= \int_0^1 \overline{f}(x)\int_0^x g(y)\dd y \dd x \\
                &= \int_0^1 \overline{f}(x)\int_0^1 g(y)\mathcal{X}_{0 < y < x < 1}\dd y \dd x, \qquad \text{where $\mathcal{X}$ is the characteristic function} \\
                &= \int_0^1 \int_0^1 \overline{f}(x)g(y)\mathcal{X}_{0<y<x<1}\dd y \dd x \\
                &= \int_0^1 g(y) \int_0^1 \overline{f}(x)\mathcal{X}_{0<y<x<1}\dd x \dd y \\
                &= \int_0^1 \qty[\int_y^1 \overline{f}(x)\dd x]g(y) \dd y \\
                &= \int_0^1 \overline{\qty[\int_y^1 f(x)\dd x]}g(y) \dd y \\
                &= \VEC{K^* f}{g}
            \end{align*}
            Thus,
            \begin{align*}
                K^*f(x) = \int_x^1 f(y)\dd y
            \end{align*}
        \end{proof}
    \item
        \emph{Show that $\norm{K} = \frac{2}{\pi}$.}
        \begin{proof}
            Define $f \in L^2([0,1])$ by
            \begin{align*}
                f(x) = \sqrt{2}\cos\frac{\pi}{2}x.
            \end{align*}
            Note $\norm{f}_{L^2} = 1$ since
            \begin{align*}
                \int_0^1 \cos^2\frac{\pi}{2}x = \int_0^1 \frac{1 + \cos \pi x}{2}\dd x = \frac{1}{2}
            \end{align*}
            Then
            \begin{align*}
                \norm{Kf}^2 &= \int_0^1\qty(\int_0^x \sqrt{2}\cos\frac{\pi}{2}y \dd y)^2 \dd x = 2\int_0^1\sin^2\frac{\pi}{2}x\dd x = \frac{4}{\pi^2}
            \end{align*}
            and thus $\norm{K} \geq \dfrac{2}{\pi}$.  
            
            Let $\norm{f} = 1$.  Then
            \begin{align*}
                \norm{Kf}^2 &= \int_0^1 (Kf)^2\dd x = \int_0^1 \qty(\int_0^x|f(y)|\dd y)^2\dd x = \int_0^1 \qty(\int_0^x\sqrt{\cos\frac{\pi}{2}y}\frac{|f(y)|}{\sqrt{\cos\frac{\pi}{2}}}\dd y)^2\dd x \\
                &\leq \int_0^1 \qty(\int_0^1 \cos \frac{\pi}{2}y\dd y)\qty(\int_0^x\frac{|f(y)|^2}{\cos\frac{\pi}{2}y})\dd x = \frac{2}{\pi}\int_0^1 \sin\qty(\frac{\pi}{2}x)\qty(\int_0^x \frac{|f(y)|^2}{\cos\frac{\pi}{2}y}\dd y)\dd x \\
                &= \frac{2}{\pi}\int_0^1 \frac{|f(y)|^2}{\cos\frac{\pi}{2}y}\qty(\int_y^1 \sin\qty(\frac{\pi}{2}x)\dd x)\dd y = \frac{4}{\pi^2}\int_0^1 |f(y)|^2 \dd y \\
                &= \frac{4}{\pi^2}\norm{f}^2 = \frac{4}{\pi^2}
            \end{align*}
            Thus $\norm{K} \leq \dfrac{2}{\pi}$.  Thus,
            \begin{align*}
                \norm{K} = \frac{2}{\pi}.
            \end{align*}
        \end{proof}
    \item
        \emph{Show that the spectral radius of $K$ is equal to zero.}
        \begin{proof}
            Let $|\lambda| > 0$ and note $L^2([0,1]) = \overline{\ran(K - \lambda I}) \oplus \ker(K - \lambda I)^*$.  Note $(K - \lambda I)^* = (K^* - \overline{\lambda}I)$ where $K^*$ is defined above.  If $(K^* - \overline{\lambda}I)f_1 = (K^* - \overline{\lambda}I)f_2$ then $\forall x \in [0,1]$,
            \begin{align*}
                \int_x^1f_1(y) \dd y &= \int_x^1 f_2(y) \dd y \\
                \implies F_1(1) - F_1(x) &= F_2(1) - F_2(x)
            \end{align*}
            where $F_1' = f_1$ and $F_2' = f_2$.  Thus $f_1 \equiv f_2$.  This shows $(K^* - \overline{\lambda}I)$ is injective, which shows $\ker(K^* - \overline{\lambda}I) = \{0\}$.  Thus,
            \begin{align*}
                L^2([0,1]) = \overline{\ran(K - \lambda I)}
            \end{align*}
            i.e.~$\ran(K - \lambda I)$ is dense in $L^2([0,1])$.  Thus showing $\ran(K - \lambda I)$ is closed will imply it is surjective.

            Let $(g_n)_n \in \ran(K - \lambda I)$ be such that $g_n \rightarrow g$.  Then  $\exists f_n$ such that $(K - \lambda I)f_n = g_n$.  Since $g_n \rightarrow g$, then by continuity (boundedness) of $(K - \lambda I)$, $\exists f$ such that $f_n \rightarrow f$.  Then
            \begin{align*}
                (K - \lambda I)f &= \int_0^x f(y) \dd y - \lambda f(x) \\
                &= \int_0^x \lim_{n \rightarrow \infty}f_n(y) \dd y - \lambda \lim_{n \rightarrow \infty}\lambda f_n(x) \\
                &= \lim_{n \rightarrow \infty} \qty[\int_0^x f_n(y) \dd y - \lambda f_n(x)], \qquad \text{by Lebesgue's Dominated Convergence Theorem} \\
                &= \lim_{n\rightarrow \infty}(K - \lambda I)f_n\\
                &= \lim_{n \rightarrow \infty}g_n \\
                &= g
            \end{align*}
            Thus $g \in \ran(K - \lambda I)$, i.e.~$\ran(K - \lambda I)$ is closed, and since $\ran(K - \lambda I)$ is dense in $L^2([0,1])$, then $(K - \lambda I)$ is surjective.  Thus $(K - \lambda I)$ is bijective, proving $\lambda \in \rho(K)$.  Thus the spectral radius of $K$, $r(K)$, is equal to $0$.
        \end{proof}
    \item
        \emph{Show that $0$ belongs to the continuous spectrum of $K$.}
        \begin{proof}
            $K$ is not onto since $K$ is the integral operator, and thus the range of $K$ is equal to the set of differentiable functions.  However, not all functions in $L^2$ are differentiable.  Thus $K$ is not onto.  However, differentiable functions are dense in $L^2$, and thus $0$ is in the continuous spectrum of $K$.
        \end{proof}
\end{enumerate}










%%%%%%%%%%%%%%%%%%%%%%%%%%%%%%%%%%%%%%
\problem{Hunter and Nachtergaele 9.8}
\emph{Define the right shift operator $S$ on $\ell^2(\mathbb{Z})$ by $$S(x)_k = x_{k-1} \qquad \text{for all }k \in \mathbb{Z},$$ where $x = (x_k)_{k=-\infty}^\infty$ is in $\ell^2(\mathbb{Z})$.  Prove the following facts.}
\begin{enumerate}[\it a)]
    \item
        \emph{The point spectrum of $S$ is empty.}
        \begin{proof}
            Suppose $\lambda$ is in the point specturm of $S$.  The for $Sx = \lambda x$ for some nonzero $x \in \ell^2(\mathbb{Z})$.  If $\lambda = 0$, the $x \equiv 0$, which is a contradiction.  If $\lambda = 1$, then $x_k = e_j$ for all $k,j \in \mathbb{Z}$, i.e. $x$ is constant.  However constant bi-infinite sequences are not in $\ell^2$ unless they are uniformly $0$.  This is a contradiction since eigenvectors are nonzero.  If $|\lambda| > 1$ and $0 < |\lambda| < 1$, then for all $k in \mathbb{Z}$, $x_k$, $Sx_k = \lambda x_{k-1}$, and thus for all $n \in \mathbb{Z}$,
            \begin{align*}
                x_k &= \lambda^{k-n} x_n, \qquad \forall n \in \mathbb{Z}.
            \end{align*}
            Thus $x_k$ can be made arbitrarily large, which is a contradiction since this is true for all $k \in \mathbb{Z}$.  Thus there are no eigenvalues of $S$ (i.e.~the point spectrum is empty).
        \end{proof}
    \item
        \emph{$\ran(\lambda I - S) = \ell^2(\mathbb{Z})$ for every $\lambda \in \Cx$ with $|\lambda| > 1$.}
        \begin{proof}
            If $\norm{(x)_n} = 1$, then $\norm{S(x)_n} = \norm{(x)_{n+1}} = \norm{(x)_n} = 1$.  Thus $\norm{S} = 1$.  Then any $\lambda \in \Cx$ such that $|\lambda| > 1$ has $\lambda \in \rho(S)$, and thus $\lambda I - S$ is bijective.  Thus $\ran (\lambda I - S) = \ell^2(\mathbb{Z})$.
        \end{proof}
    \item
        \emph{$\ran(\lambda I - S) = \ell^2(\mathbb{Z})$ for every $\lambda \in \Cx$ with $|\lambda| < 1$.}
        \begin{proof}
            Let $(y_n) \in \ell^2(\mathbb{Z})$.  Then since $\ell^2(\mathbb{Z}) \cong L^2(\mathbb{T})$, let $\mathcal{F}\ :\ \ell^2(\mathbb{Z}) \rightarrow L^2(\mathbb{T})$ be an isomorphism.  So $\exists (a_n)_n$ such that $\mathcal{F}((y_n)) = \sum_{n\in\mathbb{Z}} a_n e^{inx}$.  Then
            \begin{align*}
                \mathcal{F}(S(y)_n) = \tilde{S}\qty(\sum_{n\in\mathbb{Z}} a_n e^{inx}) = e^{ix}\sum_{n\in\mathbb{Z}}a_ne^{inx} = \sum_{n\in\mathbb{Z}}a_ne^{i(n+1)x}
            \end{align*}
            where $\tilde{S}$ is the shift operator in $L^2(\mathbb{T})$ ($\tilde{S} = \mathcal{F}\circ S$).
            Let $|\lambda| < 1$.  Then $\qty(\tilde{S} - \lambda I)g = \sum_{n\in\mathbb{Z}}a_ne^{inx}$ where $g$ is defined as
            \begin{align*}
                g = \frac{\sum_{n\in\mathbb{Z}}a_n e^{inx}}{e^{ix} - \lambda}
            \end{align*}
            since
            \begin{align*}
                \qty(\tilde{S} - \lambda I)\qty(\frac{\sum_{n\in\mathbb{Z}}a_n e^{inx}}{e^{ix} - \lambda}) = \frac{e^{ix}\sum_{n\in\mathbb{Z}}a_n e^{inx} - \lambda\sum_{n\in\mathbb{Z}}a_n e^{inx}}{e^{ix} - \lambda} = \sum_{n\in\mathbb{Z}}a_n e^{inx}
            \end{align*}
            Thus $(\tilde{S} - \lambda I)$ is surjective, which shows $(S - \lambda I)$ is surjective.
        \end{proof}
    \item
        \emph{The spectrum of $S$ consists of the unit circle $\{\lambda \in \Cx\ :\ |\lambda| = 1\}$ and is purely continuous.}
        \begin{proof}
            If $|\lambda| = 1$, then $\lambda = e^{i\theta}$ for some $\theta \in [0, 2\pi)$.  Then $\qty(\tilde{S} - \lambda I)$ is a multiplication operator, $\qty(\qty(\tilde{S} - \lambda I)f)(x) = \qty(e^{ix} - e^{i\theta})f(x)$.  Let $f \in L^2(\mathbb{T})$.  Define $g_n \in L^2(\mathbb{T})$ by
            \begin{align*}
                g_n(x) = \begin{cases}
                    \dfrac{f}{e^{ix} - e^{i\theta}} & \text{ if } x \in [0, 2\pi) \setminus [\theta - \E_n, \theta + \E_n] \\
                    0 & \text{ otherwise}
                \end{cases}
            \end{align*}
            Then define $f_n = \qty(\tilde{S} - \lambda I)g_n$.  Then
            \begin{align*}
                f_n = \begin{cases}
                    f & \text{ if } x \in [0, 2\pi) \setminus [\theta - \E_n, \theta + \E_n] \\
                    0 & \text{ otherwise}
                \end{cases}
            \end{align*}
            Let $\E_n \rightarrow 0$.  Then
            \begin{align*}
                \norm{f_n - f}^2 = \cancelto{0}{\int_0^{\theta - \E_n} |f - f|^2\dd x} + \int_{\theta-\E_n}^{\theta+\E_n} |0 - f|^2\dd x + \cancelto{0}{\int_{\theta + \E_n}^{2\pi} |f - f|^2\dd x} &= \int_{\theta-\E_n}^{\theta+\E_n}f^2\dd x \rightarrow 0 \text{ as } \E_n \rightarrow 0
            \end{align*}
            Thus $f_n \rightarrow f$ in $L^2(\mathbb{T})$.  Thus $\ran\qty(\tilde{S} - \lambda I)$ is dense in $L^2(\mathbb{T})$.  However $\qty(\tilde{S} - \lambda I)$ is not surjective since the only candidate function $g$ that would be mapped by $\qty(\tilde{S} - \lambda I)$ to $f$ is
            \begin{align*}
                g(x) = \frac{f}{e^{ix} - e^{i\theta}}
            \end{align*}
            but $g \not\in L^2(\mathbb{T})$.  Thus for $|\lambda| = 1$, $\lambda \in \sigma(\tilde{S})$ and the spectrum is purely continuous.
        \end{proof}
\end{enumerate}









%%%%%%%%%%%%%%%%%%%%%%%%%%%%%%%%%%%%%%
\problem{Hunter and Nachtergaele 9.18}
\emph{Let $P_1, \dots, P_N$ be orthogonal projections with orthogonal ranges.  Let $$A = \sum_{n=1}^N\lambda_n P_n$$ be a finite linear combination of these projections.  Let $\tilde{f}\ :\ \sigma(A) \rightarrow \Cx$ be a continuous function and define $f\ :\ \mathcal{B}(\Hilb) \rightarrow \mathcal{B}(\Hilb)$ by}
\begin{equation}
    \tag{9.23}
    f(A) = \sum_{n=1}^\infty \tilde{f}(\lambda_n)P_n.
    \label{9.23}
\end{equation}
\emph{Suppose that $A$ is a compact self-adjoint operator.  Let $f \in C(\sigma(A))$ and consider $f(A)$ defined by (\ref{9.23}).  Prove that $$\norm{f(A)} = \sup\{|\tilde{f}(\lambda_n)\ |\ n \in \mathbb{N}\}.$$  Let $(\tilde{q}_N)$ be a sequence of polynomials of degree $N$, converging uniformly to $\tilde{f}$ on $\sigma(A)$.  The existence of such a sequence is a consequence of the Weierstrass approximation theorem.  Prove that $(q_N(A))$ converges in norm, and that its limit equals $f(A)$ as defined in (\ref{9.23}).}
\begin{proof}
    The definition of operator norm guarantees that $\forall \E$, $\exists u$ with $\norm{u} = 1$ such that $\norm{f(A)}_{\text{op}} \leq \norm{f(A)u} + \E$.  But by the spectral theorem,
    \begin{align*}
        u = \sum_{n=1}^\infty \VEC{\phi_n}{u}\phi_n
    \end{align*}
    where $\{\phi_n\}$ is an orthonormal basis of eigenvectors of $f(A)$ with eigenvalues $\lambda_n$, respectively.  Thus, for any $\E > 0$,
    \begin{align*}
        \norm{f(A)}_{\text{op}} &\leq \norm{f(A)u} + \E \\
        &= \norm{f(A)\sum_{n=1}^\infty \VEC{\phi_n}{u}\phi_n} + \E
    \end{align*}
    However, since $f(A) \in \mathcal{B}(\Hilb)$, then
    \begin{align*}
        \norm{f(A)\sum_{n\in\mathbb{Z}} \VEC{\phi_n}{u}\phi_n}^2 = \norm{\sum_{n\in\mathbb{Z}} \VEC{\phi_n}{u}f(A)\phi_n}^2 &= \norm{\sum_{n\in\mathbb{Z}} \VEC{\phi_n}{u}\lambda_n\phi_n}^2 \\
        &\leq \sup_{n\in\mathbb{Z}}\left\{|\lambda_n|\right\}^2\norm{\sum_{n\in\mathbb{Z}}\VEC{\phi_n}{u}\phi_n}^2 \\
        &= \sup_{n\in\mathbb{Z}}\left\{|\lambda_n|\right\}^2\norm{u}^2, \qquad \text{by Parseval's Identity} \\
        &= \sup_{n\in\mathbb{Z}}\left\{|\lambda_n|\right\}^2
    \end{align*}
    Since $\E$ is arbitrarily small,
    \begin{align*}
        \norm{f(A)}_\text{op} \leq \sup_{n\in\mathbb{Z}}\left\{|\lambda_n|\right\}
    \end{align*}
    Also note, by the Spectral Mapping Theorem, that $\sigma(f(A)) = \tilde{f}(\sigma(A))$, and thus
    \begin{align*}
        \sup_{n\in\mathbb{Z}}\left\{\left|\lambda_n\right|\right\} = r(f(A)) \leq \norm{f(A)}_\text{op},
    \end{align*}
    which proves
    \begin{align*}
        \norm{f(A)}_\text{op} = \sup_{n\in\mathbb{Z}}\left\{|\lambda_n|\right\}
    \end{align*}
\end{proof}











\end{document}
