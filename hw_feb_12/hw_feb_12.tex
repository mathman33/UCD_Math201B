\documentclass{article} % A4 paper and 11pt font size
\setcounter{secnumdepth}{0}

\usepackage{amssymb, amsmath, amsfonts}
\usepackage{moreverb}
\usepackage{graphicx}
\usepackage{enumerate}
\usepackage{graphics}
\usepackage[margin=1.25in]{geometry}
\usepackage{color}
\usepackage{tocloft}
\renewcommand{\cftsecleader}{\cftdotfill{\cftdotsep}}
\usepackage{array}
\usepackage{float}
\usepackage{hyperref}
\usepackage{textcomp}
\usepackage[makeroom]{cancel}
\usepackage{bbold}
\usepackage{alltt}
\usepackage{physics}
\usepackage{mathtools}
\usepackage{amsthm}
\usepackage{tikz}
\usetikzlibrary{positioning}
\usetikzlibrary{arrows}
\usepackage{pgfplots}
\usepackage{bigints}
\allowdisplaybreaks
\pgfplotsset{compat=1.12}

\theoremstyle{plain}
\newtheorem*{theorem*}{Theorem}
\newtheorem{theorem}{Theorem}
\newtheorem*{lemma*}{Lemma}
\newtheorem{lemma}{Lemma}

\newenvironment{definition}[1][Definition]{\begin{trivlist}
\item[\hskip \labelsep {\bfseries #1}]}{\end{trivlist}}

\newcommand{\E}{\varepsilon}
\def\Rl{\mathbb{R}}
\def\Cx{\mathbb{C}}

\usepackage[T1]{fontenc} % Use 8-bit encoding that has 256 glyphs
\usepackage{fourier} % Use the Adobe Utopia font for the document - comment this line to return to the LaTeX default
\usepackage[english]{babel} % English language/hyphenation

\usepackage{sectsty} % Allows customizing section commands
\allsectionsfont{\centering \normalfont\scshape} % Make all sections centered, the default font and small caps

\usepackage{fancyhdr} % Custom headers and footers
\pagestyle{fancy} % Makes all pages in the document conform to the custom headers and footers
\fancyhead[L]{\bf Sam Fleischer}
\fancyhead[C]{\bf UC Davis \\ Analysis (MAT201B)} % No page header - if you want one, create it in the same way as the footers below
\fancyhead[R]{\bf Winter 2016}

\fancyfoot[L]{\bf } % Empty left footer
\fancyfoot[C]{\bf \thepage} % Empty center footer
\fancyfoot[R]{\bf } % Page numbering for right footer
\renewcommand{\headrulewidth}{0pt} % Remove header underlines
\renewcommand{\footrulewidth}{0pt} % Remove footer underlines
\setlength{\headheight}{25pt} % Customize the height of the header

\newcommand{\VEC}[2]{\left\langle #1, #2 \right\rangle}

\newcommand{\problem}[1]{
\vspace{.375cm}
\begin{minipage}{\textwidth}
    \begin{center}
        \noindent\rule{5cm}{1pt}
    \end{center}
    \section{\bf #1}
    \begin{center}
        \noindent\rule{5cm}{1pt}
    \end{center}
    \vspace{0.25cm}
\end{minipage}
}

\numberwithin{equation}{section} % Number equations within sections (i.e. 1.1, 1.2, 2.1, 2.2 instead of 1, 2, 3, 4)
\numberwithin{figure}{section} % Number figures within sections (i.e. 1.1, 1.2, 2.1, 2.2 instead of 1, 2, 3, 4)
\numberwithin{table}{section} % Number tables within sections (i.e. 1.1, 1.2, 2.1, 2.2 instead of 1, 2, 3, 4)

\setlength\parindent{0pt} % Removes all indentation from paragraphs - comment this line for an assignment with lots of text

\newcommand{\horrule}[1]{\rule{\linewidth}{#1}} % Create horizontal rule command with 1 argument of height

\title{ 
\normalfont \normalsize 
\textsc{UC Davis, Analysis (MAT201B), Winter 2016} \\ [25pt] % Your university, school and/or department name(s)
\horrule{2pt} \\[0.4cm] % Thin top horizontal rule
\Huge Homework \#4 \\ % The assignment title
\horrule{2pt} \\[0.5cm] % Thick bottom horizontal rule
}

\author{\huge Sam Fleischer} % Your name

\date{February 12, 2016} % Today's date or a custom date

\begin{document}\thispagestyle{empty}

\maketitle % Print the title

\makeatletter
\@starttoc{toc}
\makeatother

\pagebreak

\problem{Hunder and Nachtergaele 7.9}
\emph{Suppose that $u(t,x)$ is a smooth solution of the one-dimensional wave equation, $$u_{tt} - c^2 u_{xx} = 0.$$  Prove that $$\qty(u_t^2 + c^2 u_x^2)_t - \qty(2c^2 u_t u_x)_x = 0.$$  If $u(0, t) = u(1, t) = 0$ for all $t$, deduce that $$\int_0^1 |u_t(x,t)|^2 + c^2|u_x(x,t)|^2\dd x = \text{constant}.$$}

\begin{proof}
    \begin{align*}
        u_{tt} &= c^2 u_{xx} \\
        \iff 2u_tu_{tt} &= 2c^2 u_tu_{xx} \\
        \iff 2u_tu_{tt} + 2c^2 u_x u_{tx} &= 2c^2u_tu_{xx} + 2c^2 u_x u_{tx} \\
        \iff \qty(u_t^2 + c^2u_x^2)_t &= \qty(2c^2 u_t u_x)_x
    \end{align*}
    Since $u(0,t) = u(1,t) = 0$ for all $t$, then $u(0,t)_t = u(1,t) = 0$ for all $t$.  Thus
    \begin{align*}
        0 = 2c^2\qty(u_t(1,t)u_x(1,t) - u_t(0,t)u_x(0,t)) &= (2c^2 u_tu_x)\Big|_{x=0}^1 \\
        &= \int_0^1\qty(2c^2 u_t u_x)_x \dd x \\
        &= \int_0^1\qty(u_t^2 + c^2 u_x^2)_t\dd x \\
        &= \frac{\dd}{\dd t}\int_0^1\qty(u_t^2 + c^2 u_x^2)\dd x \\
        \iff \int_0^1 (u_t^2 + c^2 u_x^2)\dd x &= \text{constant}.
    \end{align*}
\end{proof}

\problem{Hunder and Nachtergaele 7.10}
\emph{Show that $$u(x,t) = f(x + ct) + g(x - ct)$$ is a solution of the one-dimensional wave equation $$u_{tt} - c^2u_{xx} = 0,$$ for arbitrary functions $f$ and $g$.  This solution is called \emph{d'Alembert's solution}.} \\

\begin{align*}
    u(x,t) &= f(x + ct) + g(x - ct) \\
    \implies u_t(x,t) &= c (f'(x + ct) - g'(x - ct)) \\
    \implies u_{tt}(x,t) &= c^2 (f''(x + ct) + g''(x - ct)) \\
\end{align*}
Also,
\begin{align*}
    u(x,t) &= f(x + ct) + g(x - ct) \\
    \implies u_x(x,t) &= f'(x + ct) + g'(x - ct) \\
    \implies u_{xx}(x,t) &= f''(x + ct) + g''(x - ct) \\
\end{align*}
Thus,
\begin{align*}
    u_{tt}(x,t) &= c^2 (f''(x + ct) + g''(x - ct)) = c^2 u_{xx}(x,t)
\end{align*}

\problem{Hunder and Nachtergaele 7.14}
\emph{Consider the logistic map $$x_{n+1} = 4\mu x_n(1 - x_n),$$ where $x_n \in [0,1]$, and $\mu = 1$.  Show that the solutions may be written as $x_n = \sin^2\theta_n$ where $\theta_n \in \mathbb{T}$, and $$\theta_{n+1} = 2\theta_n.$$  What can you say about the orbits of the logistic map, the exitence of an invariant measure, and the validity of an ergodic theorem?} \\

Let $x_n = \sin^2 \theta_n$ and $\theta_{n+1} = 2\theta_n$.  Then
\begin{align*}
    \theta_{n+1} &= 2\theta_n \\
    \implies \sin^2(\theta_{n+1}) &= \sin^2(2\theta_n) \\
    \implies x_{n+1} &= 4\sin^2\theta_n\cos^2\theta_n \\
    \implies x_{n+1} &= 4\sin^2\theta_n(1 - \sin^2\theta_n) \\
    \implies x_{n+1} &= 4x_n(1 - x_n)
\end{align*}
Thus $x_n = \sin^2 \theta_n$, where $\theta_{n+1} = 2\theta_n$,
satisfies the logistic map.

FIrst, we show the map $T: [0,1] \rightarrow [0,1]$ by $T\theta = 2\theta \mod 1$ preserves the Lebesgue measure $\mathcal{L}$ for all Borel sets on $[0,1]$.  Consider the $\sigma$-algebra
\begin{align*}
    \mathcal{F} = \left\{A\ :\ \mathcal{L}\qty(T^{-1}A) = \mathcal{L}\qty(A)\right\}
\end{align*}
It suffices to show all intervals are contained in $\mathcal{F}$ since the Borel $\sigma$-algebra is the smallest $\sigma$-algebra to contain the intervals.  However, for any interval $[a,b] \subset [0,1]$, $T\qty[\frac{a}{2}, \frac{b}{2}] = [a, b]$ and $T\qty[\frac{3a}{2}, \frac{3b}{2}] = [a,b]$ and no other intervals are mapped to $[a,b]$.  Also,
\begin{align*}
    \mathcal{L}\qty(T^{-1}[a,b]) = \mathcal{L}\qty(\qty[\frac{a}{2}, \frac{b}{2}] \cup \qty[\frac{3a}{2}, \frac{3b}{2}]) = \mathcal{L}\qty(\qty[\frac{a}{2}, \frac{b}{2}]) + \mathcal{L}\qty(\qty[\frac{3a}{2}, \frac{3b}{2}]) = \frac{b-a}{2} + \frac{b-a}{2} = b - a = \mathcal{L}\qty([a,b])
\end{align*}
Thus all intervals are contained in $\mathcal{F}$ and thus $T$ preserves the Lebesgue measure for any Borel set on $[0,1]$.

\problem{Hunder and Nachtergaele 7.15}
\emph{Consider the dynamical system on $\mathbb{T}$, $$x_{n+1} = \alpha x_n \mod 1,$$ where $\alpha = (1 + \sqrt{5})/2$ is the golden ration.  Show that the orbit with initial value $x_0 = 1$ is not equidistributed on the circle, meaning that it does not satisy (7.39).}

\emph{HINT.  Show that $$u_n = \qty(\frac{1 + \sqrt{5}}{2})^n + \qty(\frac{1 - \sqrt{5}}{2})^n$$ satisfies the difference equation $$u_{n+1} = u_n + u_{n-1}$$ and hence s an integer for every $n \in \mathbb{N}$.  Then use the fact that $$\qty(\frac{1 - \sqrt{5}}{2})^n \rightarrow 0\ \ \text{ as } n \rightarrow \infty.$$}

Let $\phi^+ = \frac{1 + \sqrt{5}}{2}$ and $\phi^- = \frac{1- \sqrt{5}}{2}$.  Clearly the dynamical system is not equidistributed on $[0,1]$ since if $x_0 = 1$, then $x_1 = \phi^+ \mod 1 = -\phi^-$ and $x_2 = -\phi^-\cdot \phi^+ = 1$.  Thus the system has orbit of length 2 and any finite orbit cannot be equidistributed in an interval. \\

However, the hint is implying the writer intended to ask us to show that the following sequence is not equidistributed on $[0,1]$.
\begin{align*}
    \left\{\qty(\phi^+)^n\mod 1\right\}_{n=0}^\infty
\end{align*}
Note that this is not a dynamical system since it is not recursive.  Let $(u_n)_n$ be a sequence defined by
\begin{align*}
    u_n = \qty(\phi^+)^n + \qty(\phi^-)^n
\end{align*}
Note that this sequence satisfies the recursion relation
\begin{align*}
    u_{n+1} = u_n + u_{n-1}
\end{align*}
since
\begin{align*}
    u_n + u_{n-1} &= \qty(\phi^+)^n + \qty(\phi^-)^n + \qty(\phi^+)^{n-1} + \qty(\phi^-)^{n-1} \\
    &= \qty(\phi^+)^{n-1}\qty[1 + \phi^+] + \qty(\phi^-)^{n-1}\qty[1 + \phi^-] \\
    &= \qty(\phi^+)^{n-1}\qty[\frac{3 + \sqrt{5}}{2}] + \qty(\phi^-)^{n-1}\qty[\frac{3 - \sqrt{5}}{2}] \\
    &= \qty(\phi^+)^{n+1} + \qty(\phi^-)^{n+1} \\
    &= u_{n+1}
\end{align*}
Since $u_0 = 2$ and $u_1 = 1$, then $u_n \in \mathbb{N}$ for all $n$.  Then note that since $\left|\phi^-\right| < 1$, then $\qty(\phi^-)^n \rightarrow 0$.  Thus for any $\E$, there exists $N_\E$ such that, $\qty(\phi^+)^n \mod 1 \in (0, \E) \cup (1 - \E, 1]$ for all $n \geq N_\E$.  Thus $\#\{u_n\ |\ u_n \in [\E, 1 - \E]\} \leq N_\E$ and so the sequence is not equidistributed in $[0,1]$.

\problem{Hunder and Nachtergaele 7.17}
\emph{Let $B_n$ and $V_n$ be defined in (7.46) and (7.47).  Prove that $\bigcup_{n=0}^N B_n$ is an orthonormal basis of $V_N$.}

\emph{HINT.  Prove that the set is orthonormal and count its elements.} \\

Let $V_n$ be the finite dimensional subspace of $L^2[0,1]$
\begin{align*}
    V_n = \left\{f\ :\ f \text{ is constant on } \left[\frac{k}{2^n}, \frac{k+1}{2^n}\right) \text{ for } k = 0, \dots, 2^n - 1\right\}
\end{align*}
and define subsets $B_n$ of $V_n$ by
\begin{align*}
    B_0 = \{\phi_{0,0}\},\qquad B_{n+1} = \{\psi_{n,k}\ |\ k = 0, 1, \dots, 2^n - 1\}
\end{align*}
where
\begin{align*}
    \phi_{n,k}(x) = 2^{\frac{n}{2}}\phi(2^n - k), \qquad \psi_{n,k}(x) = 2^{\frac{n}{2}}\psi(2^n x - k)
\end{align*}
and
\begin{align*}
    \phi(x) = \begin{cases}
        1 & \text{ if } 0 \leq x < 1, \\
        0 & \text{ otherwise}
    \end{cases} \qquad \text{and} \qquad \psi(x) = \begin{cases}
        1 & \text{ if } 0 \leq x \leq \frac{1}{2} \\
        -1 & \text{ if } \frac{1}{2} \leq x < 1 \\
        0 & \text{ otherwise}
    \end{cases}
\end{align*}

First define $e_k$, $k = 0, 1, \dots, 2^N - 1$ by
\begin{align*}
    e_k(x) = \begin{cases}
        1 & \text{ if } x \in \left[\frac{k}{2^n}, \frac{k+1}{2^n}\right) \\
        0 & \text{ otherwise }
    \end{cases}
\end{align*}
and note $E = \big\{e_k\big\}_{k=0}^{2^N-1}$ is a basis since if $f \in V_N$ then $f$ is constant on each interval and hence can be written as a linear combination of elements of $E$.  Thus $\dim(V_N) = 2^N$.

Next define $B = \bigcup_{n=0}^N B_N$ and note $\#B = 2^N$.  Next we show that $\#B$ is an orthonormal set.  Fix $n$ and $k \in \mathbb{Z}\cap[0, 2^n -1]$.  Then
\begin{align*}
    \VEC{\psi_{n,k}}{\psi_{n,k}}_{L^2} = \int_0^1 \psi_{n,k}^2 \dd x = \int_{\frac{k}{2^n}}^{\frac{k+1}{2^n}} \qty(2^{\frac{n}{2}})^2 = 2^n\qty(\frac{1}{2^n}) = 1
\end{align*}
Also, the support of $\psi_{n,k_1}$ is disjoint from the support of $\psi_{n,k_2}$, thus
\begin{align*}
    \VEC{\psi_{n,k_1}}{\psi_{n,k_2}} = \int_0^1 \psi_{n, k_1}\psi_{n,k_2}\dd x = 0
\end{align*}
For $n_2 > n_1$, then the support of $\psi_{n_2, k_2}$ is either (i) disjoint from the support of $\psi_{n_1, k_1}$ or (ii) contained in either the left or right halves of the support of $\psi_{n_1, k_1}$.  If (i), then their inner product is zero since their supports are disjoint.  If (ii), then without loss of generality suppose the support of $\psi_{n_2,k_2}$ is contained in the right half of the support of $\psi_{n_1,k_1}$.  Then
\begin{align*}
    \VEC{\psi_{n_1, k_1}}{\psi_{n_2, k_2}} = \int_0^1 \psi_{n_1,k_1}\psi_{n_2,k_2} \dd x = \int_{\frac{k_2}{2^{n_2}}}^{\frac{k_2+1}{2^{n_2}}}2^{\frac{n_1}{2}}\psi_{n_2,k_2}\dd x = 2^{\frac{n_1}{2}}\int_{\frac{2k_2}{2^{n_2 + 1}}}^{\frac{2k_2 + 1}{2^{n_2 + 1}}} 2^{\frac{n_2}{2}} \dd x + 2^{\frac{n_1}{2}}\int_{\frac{2k_2 + 1}{2^{n_2 + 1}}}^{\frac{2k_2 + 2}{2^{n_2 + 1}}} \qty(-2^{\frac{n_2}{2}}) \dd x = 0
\end{align*}
Thus $B$ is an orthonormal set.  Also, $B \subset V_N$ since each $b \in B$ is constant on each of the required intervals.  Thus $B$ is an orthonormal basis of $V_N$.

\problem{Hunder and Nachtergaele 7.18}
\emph{Suppose that $B = \{e_n(x)\}_{n=1}^\infty$ is an orthonormal basis for $L^2([0,1])$.  Prove the following:}
\begin{enumerate}[ (a)]
    \item
        \emph{For any $a \in \Rl$, the set $B_a = \left\{e_n(x - a)\right\}_{n=1}^\infty$ is an orthonormal basis for $L^2([a, a+1])$.} \\

        Since $e_n(x-a)$ is a horizontal shift then
        \begin{align*}
            \VEC{e_n(x-a)}{e_n(x-a)}_{L^2([a, a+1])} = \int_a^{a+1}(e_n(x-a))^2\dd x = \int_0^1 (e_n(x))^2 \dd x = \VEC{e_n(x)}{e_n(x)}_{L^2([0, 1])} = 1
        \end{align*}
        Similarly,
        \begin{align*}
            \VEC{e_n(x-a)}{e_m(x-a)}_{L^2([a, a+1])} = \int_a^{a+1}e_n(x-a)e_m(x-a)\dd x = \int_0^1e_n(x)e_m(x) = \VEC{e_n(x)}{e_m(x)}_{L^2([0,1])} = 0
        \end{align*}
        Thus $\left\{e_n(x-a)\right\}_{n=1}^\infty$ is orthonormal.  Suppose $\VEC{f(x-a)}{e_n(x-a)}_{L^2([a, a+1])} = 0$ for all $n = 1, 2, \dots$.  Then
        \begin{align*}
            0 = \int_a^{a+1}f(x-a)e_n(x-a)\dd x = \int_0^1 f(x)e_n(x)\dd x = \VEC{f(x)}{e_n(x)}_{L^2([0,1])}
        \end{align*}
        for all $n$.  Since $\left\{e_n(x)\right\}_{n=1}^\infty$ is a basis of $L^2([0,1])$, then $f(x) \equiv 0$.  Thus $f(x-a) \equiv 0$, proving $\left\{e_n(x-a)\right\}_{n=1}^\infty$ is a basis of $L^2([a, a+1])$.
    \item
        \emph{For any $c > 0$, the set $B^C = \left\{\sqrt{c}e_n(cx)\right\}_{n=1}^\infty$ is an orthonormal basis for $L^2\qty(\qty[0, c^{-1}])$.} \\

        Since $e_n(cx)$ is a horizontal compression,
        \begin{align*}
            \VEC{\sqrt{c}e_n(cx)}{\sqrt{c}e_n(cx)}_{L^2\qty(\qty[0, c^{-1}])} = c\int_0^{c^{-1}} (e_n(cx))^2 \dd x = c\cdot c^{-1}\int_0^1 (e_n(x))^2\dd x = \VEC{e_n(x)}{e_n(x)}_{L^2([0,1])} = 1
        \end{align*}
        Similarly,
        \begin{align*}
            \VEC{\sqrt{c}e_n(cx)}{\sqrt{c}e_m(cx)}_{L^2\qty(\qty[0, c^{-1}])} = c\int_0^{c^{-1}} e_n(cx)e_m(cx) \dd x = c\cdot c^{-1}\int_0^1 e_n(x)e_m(x)\dd x = \VEC{e_n(x)}{e_m(x)}_{L^2([0,1])} = 0
        \end{align*}
        Thus $B^C$ is orthonormal.  Suppose $\VEC{\sqrt{c}f(cx)}{\sqrt{c}e_n(cx)} = 0$ for all $n = 1, 2, \dots$.  Then
        \begin{align*}
            0 = c\int_0^{c^{-1}}f(cx)e_n(cx)\dd x = c\cdot c^{-1}\int_0^1 f(x)e_n(x)\dd x = \VEC{f(x)}{e_n(x)}_{L^2([0,1])}
        \end{align*}
        for all $n$.  Since $\left\{e_n(x)\right\}_{n=1}^\infty$ is a basis of $L^2([0,1])$, then $f(x) \equiv 0$.  Thus $\sqrt{c}f(cx) \equiv 0$, proving $B^C$ is a basis of $L^2\qty(\qty[0, c^{-1}])$.
    \item
        \emph{With the convention that functions are extended to a larger domain than their original domain by setting them equal to $0$, prove that $B \cup B_1$ is an orthonormal basis for $L^2([0,2])$.} \\

        We know $B$ is an orthonormal basis of $L^2([0,1])$ and by part (a), $B_1$ is an orthonormal basis of $L^2([1,2])$.  Then let $e_n \in B$ and $e_m \in B_1$.  Then $e_n(x) = 0$ for $x \in (1,2 )$ and $e_m(x) = 0$ for $x \in (0, 1)$.  Thus
        \begin{align*}
            \VEC{e_n}{e_m}_{L^2([0,2])} = \int_0^2 e_n e_m \dd x = \int_0^1 e_n e_m \dd x + \int_1^2 e_n e_m \dd x = 0
        \end{align*}
        Thus $B \cup B_1$ is an orthonormal set.  Let $f \in L^2([0,2])$.  Then $f = f_1 + f_2$ where
        \begin{align*}
            f_1(x) = \begin{cases}
                f(x) & \text{ if } x \in [0,1) \\
                0 & \text{ if } x \in [1, 2]
            \end{cases} \qquad \text{and} \qquad f_2(x) = \begin{cases}
                0 & \text{ if } x \in [0, 1) \\
                f(x) & \text{ if } x \in [1, 2]
            \end{cases}
        \end{align*}
        Thus $f_1 \in L^2([0,1])$ and $f_2 \in L^2([1,2])$, and so
        \begin{align*}
            f_1(x) = \sum_{i=1}^\infty \alpha_i e_{n_i}(x) \qquad \text{and} \qquad f_2(x) = \sum_{j=1}^\infty \beta_j e_{n_j}(x-1)
        \end{align*}
        which implies
        \begin{align*}
            f = f_1 + f_2 = \sum_{i=1}^\infty \alpha_i e_{n_i}(x) + \sum_{j=1}^\infty \beta_j e_{n_j}(x-1)
        \end{align*}
        This shows $f$ can be written as a linear combination of elements of $B \cup B_1$ and thus $B \cup B_1$ is a basis of $L^2([0,2])$.
    \item
        \emph{Prove that $\bigcup_{k\in\mathbb{Z}} B_k$ is an orthonormal basis for $L^2(\Rl)$.} \\

        By part (a), each $B_k$ is an orthonormal basis of $L^2([k, k+1])$.  Also, for $k_1 < k_2$, let $e_n(x - k_1) \in B_{k_1}$ and $e_m(x - k_2) \in B_{k_2}$.  Then
        \begin{align*}
            \VEC{e_n(x-k_1)}{e_m(x-k_2)}_{L^2(\Rl)} &= \int_\Rl e_n(x-k_1)e_m(x-k_2)\dd x \\
            &= \cancelto{0}{\int_{-\infty}^{k_1} e_n(x-k_1)e_m(x-k_2)\dd x} + \int_{k_1}^{k_1+1} e_n(x-k_1)e_m(x-k_2)\dd x \\
            &\qquad\qquad + \cancelto{0}{\int_{k_1+1}^{k_2} e_n(x-k_1)e_m(x-k_2)\dd x} + \int_{k_2}^{k_2+1}e_n(x-k_1)e_m(x-k_2)\dd x \\
            &\qquad\qquad + \cancelto{0}{\int_{k_2+1}^\infty e_n(x-k_1)e_m(x-k_2)\dd x} \\
            &= \int_{k_1}^{k_1+1}e_n(x-k_1)\cdot 0 \dd x + \int_{k_2}^{k_2+1}0\cdot e_m(x-k_2)\dd x \\
            &= 0
        \end{align*}
        Thus $\bigcup_{k\in\mathbb{Z}}$ is orthonormal.  Then let $f \in L^2(\Rl)$.  Then
        \begin{align*}
            f = \sum_{k\in\mathbb{Z}}f_k
        \end{align*}
        where $f_k \in B_k$ and for each $k$,
        \begin{align*}
            f_k(x) = \begin{cases}
                f(x) & \text{ if } x \in [k, k+1) \\
                0 & \text{ otherwise}
            \end{cases}
        \end{align*}
        Then since $f_k \in B_k$,
        \begin{align*}
            f_k = \sum_{i_k = 1}^\infty \alpha_{i_k} e_{n_{i_k}}(x - k)
        \end{align*}
        since $\left\{e_n(x-k)\right\}_{n=1}^\infty$ is a basis for $B_k$.  Thus,
        \begin{align*}
            f = \sum_{k \in\mathbb{Z}}f_k = \sum_{k\in\mathbb{Z}}\qty[\sum_{i_k = 1}^\infty \qty(\alpha_{i_k} e_{n_{i_k}}(x-k))]
        \end{align*}
        and thus $f$ can be written as a linear combination of elements of $\bigcup_{k\in\mathbb{Z}}B_k$.  Thus $\bigcup_{k\in\mathbb{Z}}$ is a basis of $L^2(\Rl)$.
\end{enumerate}


\end{document}
