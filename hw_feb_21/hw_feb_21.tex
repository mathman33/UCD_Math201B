\documentclass{article} % A4 paper and 11pt font size
\setcounter{secnumdepth}{0}

\usepackage{amssymb, amsmath, amsfonts}
\usepackage{moreverb}
\usepackage{graphicx}
\usepackage{enumerate}
\usepackage{graphics}
\usepackage[margin=1.25in]{geometry}
\usepackage{color}
\usepackage{tocloft}
\renewcommand{\cftsecleader}{\cftdotfill{\cftdotsep}}
\usepackage{array}
\usepackage{float}
\usepackage{hyperref}
\usepackage{textcomp}
\usepackage[makeroom]{cancel}
\usepackage{bbold}
\usepackage{alltt}
\usepackage{physics}
\usepackage{mathtools}
\usepackage{amsthm}
\usepackage{tikz}
\usetikzlibrary{positioning}
\usetikzlibrary{arrows}
\usepackage{pgfplots}
\usepackage{bigints}
\allowdisplaybreaks
\pgfplotsset{compat=1.12}

\theoremstyle{plain}
\newtheorem*{theorem*}{Theorem}
\newtheorem{theorem}{Theorem}
\newtheorem*{lemma*}{Lemma}
\newtheorem{lemma}{Lemma}

\newenvironment{definition}[1][Definition]{\begin{trivlist}
\item[\hskip \labelsep {\bfseries #1}]}{\end{trivlist}}

\newcommand{\E}{\varepsilon}
\def\Rl{\mathbb{R}}
\def\Cx{\mathbb{C}}

\usepackage[T1]{fontenc} % Use 8-bit encoding that has 256 glyphs
\usepackage{fourier} % Use the Adobe Utopia font for the document - comment this line to return to the LaTeX default
\usepackage[english]{babel} % English language/hyphenation

\usepackage{sectsty} % Allows customizing section commands
\allsectionsfont{\centering \normalfont\scshape} % Make all sections centered, the default font and small caps

\usepackage{fancyhdr} % Custom headers and footers
\pagestyle{fancy} % Makes all pages in the document conform to the custom headers and footers
\fancyhead[L]{\bf Sam Fleischer}
\fancyhead[C]{\bf UC Davis \\ Analysis (MAT201B)} % No page header - if you want one, create it in the same way as the footers below
\fancyhead[R]{\bf Winter 2016}

\fancyfoot[L]{\bf } % Empty left footer
\fancyfoot[C]{\bf \thepage} % Empty center footer
\fancyfoot[R]{\bf } % Page numbering for right footer
\renewcommand{\headrulewidth}{0pt} % Remove header underlines
\renewcommand{\footrulewidth}{0pt} % Remove footer underlines
\setlength{\headheight}{25pt} % Customize the height of the header

\newcommand{\VEC}[2]{\left\langle #1, #2 \right\rangle}
\newcommand{\ran}{\text{\rm ran }}

\newcommand{\problem}[1]{
\vspace{.375cm}
\begin{minipage}{\textwidth}
    \begin{center}
        \noindent\rule{5cm}{1pt}
    \end{center}
    \section{\bf #1}
    \begin{center}
        \noindent\rule{5cm}{1pt}
    \end{center}
    \vspace{0.25cm}
\end{minipage}
}

\numberwithin{equation}{section} % Number equations within sections (i.e. 1.1, 1.2, 2.1, 2.2 instead of 1, 2, 3, 4)
\numberwithin{figure}{section} % Number figures within sections (i.e. 1.1, 1.2, 2.1, 2.2 instead of 1, 2, 3, 4)
\numberwithin{table}{section} % Number tables within sections (i.e. 1.1, 1.2, 2.1, 2.2 instead of 1, 2, 3, 4)

\setlength\parindent{0pt} % Removes all indentation from paragraphs - comment this line for an assignment with lots of text

\newcommand{\horrule}[1]{\rule{\linewidth}{#1}} % Create horizontal rule command with 1 argument of height

\title{ 
\normalfont \normalsize 
\textsc{UC Davis, Analysis (MAT201B), Winter 2016} \\ [25pt] % Your university, school and/or department name(s)
\horrule{2pt} \\[0.4cm] % Thin top horizontal rule
\Huge Homework \#5 \\ % The assignment title
\horrule{2pt} \\[0.5cm] % Thick bottom horizontal rule
}

\author{\huge Sam Fleischer} % Your name

\date{February 21, 2016} % Today's date or a custom date

\begin{document}\thispagestyle{empty}

\maketitle % Print the title

\makeatletter
\@starttoc{toc}
\makeatother

\pagebreak

%%%%%%%%%%%%%%%%%%%%%%%%%%%%%%%%%%%%%%
\problem{Hunter and Nachtergaele 8.2}
\emph{If $\mathcal{H} = \mathcal{M} \oplus \mathcal{N}$ is an orthogonal direct sum, show that $\mathcal{M}^\perp = \mathcal{N}$ and $\mathcal{N}^\perp = \mathcal{M}$.}

\begin{proof}
    Suppose $x \in \mathcal{M}^\perp$.  Then $x \in \mathcal{H} \implies \exists!y, z$ such that $x = y + z$ where $y \in \mathcal{M}$ and $z \in \mathcal{N}$.  Then
    \begin{align*}
        \VEC{x}{x} = \cancelto{0}{\VEC{x}{y}} + \VEC{x}{z} = \VEC{x}{z} \implies \VEC{x}{x-z} = 0 \implies x = z
    \end{align*}
    which shows $x \in \mathcal{N}$, i.e.~$\mathcal{M}^\perp \subset \mathcal{N}$.

    Now suppose $x \not\in \mathcal{M}^\perp$.  Then $x \in \mathcal{M}$ and $x \neq 0$.  Thus $x \not\in \mathcal{N}$ since a direct sum implies $\mathcal{N} \cap \mathcal{M} = \{0\}$.  Thus $\mathcal{N} \subset \mathcal{M}^\perp \subset \mathcal{N} \implies \mathcal{N} = \mathcal{M}^\perp$.

    Switching $\mathcal{N}$ and $\mathcal{M}$ shows $\mathcal{M} = \mathcal{N}^\perp$.
\end{proof}









%%%%%%%%%%%%%%%%%%%%%%%%%%%%%%%%%%%%%%
\problem{Hunter and Nachtergaele 8.3}
\emph{Let $\mathcal{M}$, $\mathcal{N}$ be closed subspaces of a Hilbert space $\mathcal{H}$ and $P$, $Q$ the orthogonal projections with $\ran P = \mathcal{M}$, $\ran Q = \mathcal{N}$.  Prove that the following conditions are equivalent: (a) $\mathcal{M} \subset \mathcal{N}$; (b) $QP = P$; (c) $PQ = P$; (d) $\norm{Px} \leq \norm{Qx}$ for all $x \in \mathcal{H}$; (e) $\VEC{x}{Px} \leq \VEC{x}{Qx}$ for all $x \in \mathcal{H}$.}

\begin{proof}
    We will show \emph{(a) $\rightarrow$ (b) $\rightarrow$ (c) $\rightarrow$ (d) $\rightarrow$ (e) $\rightarrow$ (a)}, which proves the statements' equivalence.
    \begin{enumerate}
        \item[\it (a) $\rightarrow$ (b). ]
            Let $\mathcal{M} \subset \mathcal{N}$ and let $x \in \mathcal{H}$.  Then $Px \in \mathcal{M} \subset \mathcal{N} \implies Q(Px) = Px \implies QP = P$.
        \item[\it (b) $\rightarrow$ (c). ]
            Let $QP = P$.  Then
            \begin{align*}
                \VEC{x}{Py} = \VEC{Px}{y} = \VEC{QPx}{y} = \VEC{Px}{Qy} = \VEC{x}{PQy} \implies \VEC{x}{Py - PQy} = 0 \qquad \forall x, y \in \mathcal{H}.
            \end{align*}
            Thus $Py - PQy = 0$ for all $y \in \mathcal{H}$, i.e.~$PQ = P$.
        \item[\it (c) $\rightarrow$ (d). ]
            Let $PQ = P$.  First note $\norm{Px} \leq \norm{x}$ for all $x \in \mathcal{H}$ because
            \begin{align*}
                \norm{Px} = \frac{\norm{Px}^2}{\norm{Px}} = \frac{\VEC{Px}{Px}}{\norm{Px}} = \frac{\VEC{x}{P^2x}}{\norm{Px}} = \frac{\VEC{x}{Px}}{\norm{Px}} \leq \norm{x}
            \end{align*}
            by the Cauchy-Schwarz inequality.  Thus,
            \begin{align*}
                \norm{Px} = \norm{PQx} = \norm{P(Qx)} \leq \norm{Qx} \qquad \forall x \in \mathcal{H}
            \end{align*}
        \item[\it (d) $\rightarrow$ (e). ]
            Let $\norm{Px} \leq \norm{Qx}$.
            \begin{align*}
                \norm{Px}^2 \leq \norm{Qx}^2 \implies \VEC{Px}{Px} \leq \VEC{Qx}{Qx} \implies \VEC{x}{P^2x} \leq \VEC{x}{Q^2x} \implies \VEC{x}{Px} \leq \VEC{x}{Qx} \qquad \forall x \in \mathcal{H}
            \end{align*}
        \item[\it (e) $\rightarrow$ (a). ]
            Let $\VEC{x}{Px} \leq \VEC{x}{Qx}$ and suppose $x \in M$.  Then $Px = x$.  Then $\norm{x}^2 = \VEC{x}{x} = \VEC{x}{Px} \leq \VEC{x}{Qx}$.  However, since $\norm{Qx} \leq \norm{x}$ for all $x$, then $\VEC{x}{Qx} \leq \norm{x}\norm{Qx} \leq \norm{x}^2$, which shows $\VEC{x}{Qx} = \norm{x}^2$.  Thus $Qx = x$, which shows $x \in \mathcal{N}$, proving $\mathcal{M} \subset \mathcal{N}$.
    \end{enumerate}
\end{proof}










%%%%%%%%%%%%%%%%%%%%%%%%%%%%%%%%%%%%%%
\problem{Hunter and Nachtergaele 8.4}
\emph{Suppose that $(P_n)$ is a sequence of orthogonal projections on a Hilbert space $\mathcal{H}$ such that $$\ran P_{n+1} \supset \ran P_n, \qquad \bigcup_{n=1}^\infty \ran P_n = \mathcal{H}.$$  Prove that $(P_n)$ converges strongly to the identity operator $I$ as $n \rightarrow \infty$.  Show that $(P_n)$ does not converge to the identity operator with respect to the operator norm unless $P_n = I$ for all sufficiently large $n$.} \\

\begin{proof}
    Let $x \in \mathcal{H}$.  Then $\exists N$ such that $x \in \ran P_N$.  Thus $x \in \ran P_n$ for all $n \geq N$.  Since each $P_n$ is an orthogonal projection, then $x = P_n x$ for all $n \geq N$.  Thus
    \begin{align*}
        \lim_{n\rightarrow \infty}P_n x = \lim_{n \rightarrow \infty} x = x = I x
    \end{align*}
    where $I$ is the identity operator.  Thus $(P_n)$ converges strongly to $I$.

    If $P_n = I$ for all sufficiently large $n$, then obviously
    \begin{align*}
        \lim_{n\rightarrow \infty} \norm{P_n} = \lim_{n\rightarrow \infty}\norm{I} = \norm{I}
    \end{align*}
\end{proof}









%%%%%%%%%%%%%%%%%%%%%%%%%%%%%%%%%%%%%%
\problem{Hunter and Nachtergaele 8.6}
\emph{Show that a linear operator $U\ :\ \mathcal{H}_1 \rightarrow \mathcal{H}_2$ is unitary if and only if it is an isometric isomorphism of normed linear spaces.  Show that an invertible linear map is unitary if and only if its inverse is.} \\

\begin{proof}
    Let $U\ :\ \mathcal{H}_1 \rightarrow \mathcal{H}_2$ be unitary.  Then $U$ is invertible and $\VEC{Ux}{Uy}_{\mathcal{H}_2} = \VEC{x}{y}_{\mathcal{H}_1}$ for all $x, y \in \mathcal{H}_1$.  Thus
    \begin{align*}
        \norm{Ux}_{\mathcal{H}_2}^2 = \VEC{Ux}{Ux}_{\mathcal{H}_2} = \VEC{x}{x}_{\mathcal{H}_1} = \norm{x}_{\mathcal{H}_1}^2
    \end{align*}
    Thus $U$ preserves norms and is thus an isometric isomorphism.

    Now suppose $U$ is an isometric isomorphism.  Then $\norm{Ux}_{\mathcal{H}_2} = \norm{x}_{\mathcal{H}_1}$ for all $x \in \mathcal{H}_1$.
\end{proof}










%%%%%%%%%%%%%%%%%%%%%%%%%%%%%%%%%%%%%%
\problem{Hunter and Nachtergaele 8.7}
\emph{If $\phi_y$ is the bounded linear functional defined in (8.5),}
\begin{equation}
    \tag{\it 8.5}
    \phi_y(x) = \VEC{y}{x}
\end{equation}
\emph{prove that $\norm{\phi_y} = \norm{y}$.} \\

\begin{proof}
    First we prove $\norm{\phi_y}$ is bounded above by $\norm{y}$.
    \begin{align*}
        \norm{\phi_y} = \sup_{\norm{x} = 1}\norm{\phi_y(x)} = \sup_{\norm{x} = 1}\VEC{y}{x} \leq \sup_{\norm{x} = 1}\norm{y}\norm{x} = \norm{y}
    \end{align*}
    Next, consider $x = \frac{y}{\norm{y}}$ (note $\norm{x} = 1$):
    \begin{align*}
        \norm{\phi_y(x)} = \norm{\VEC{y}{x}} = \norm{\frac{\VEC{y}{y}}{\norm{y}}} = \norm{\frac{\norm{y}^2}{\norm{y}}} = \norm{y}
    \end{align*}
    and thus $\norm{\phi_y} \geq \norm{y}$, which proves $\norm{\phi_y} = \norm{y}$.
\end{proof}










%%%%%%%%%%%%%%%%%%%%%%%%%%%%%%%%%%%%%%
\problem{Hunter and Nachtergaele 8.8}
\emph{Prove that $\mathcal{H}^*$ is a Hilbert space with the inner product defined by $$ \VEC{\phi_x}{\phi_y}_{\mathcal{H}^*} = \VEC{y}{x}_\mathcal{H}.$$}

\begin{proof}
    First note that for $\phi_{y_1}$, $\phi_{y_2} \in \mathcal{H}^*$, $\norm{\phi_{y_1} + \phi_{y_2}}_{\mathcal{H}^*} = \norm{y_1 + y_2}_{\mathcal{H}}$ where $y_1$ and $y_2$ are the associated vectors in $\mathcal{H}$ guaranteed in the Riesz Representation Theorem.  This is true because $\phi_{y_1}(x) = \VEC{y_1}{x}_{\mathcal{H}}$ and $\phi_{y_2}(x) = \VEC{y_2}{x}_{\mathcal{H}}$ imply
    \begin{align*}
         \phi_{y_1 + y_2}(x) = \VEC{y_1 + y_2}{x}_{\mathcal{H}} = \VEC{y_1}{x}_{\mathcal{H}} + \VEC{y_2}{x}_{\mathcal{H}} = \phi_{y_1}(x) + \phi_{y_2}(x)
    \end{align*}
    and since $\norm{\phi_{y_1 + y_2}}_{\mathcal{H}^*} = \norm{y_1 + y_2}_{\mathcal{H}}$ by (Hunter and Nachtergaele 8.7), then $\norm{\phi_{y_1} + \phi_{y_2}}_{\mathcal{H}^*} = \norm{y_1 + y_2}_{\mathcal{H}}$.

    Let $(\phi_n)$ be a Cauchy sequence in $\mathcal{H}^*$.  Then $\forall \E > 0$, $\exists N$ such that $\norm{\phi_m - \phi_n}_{\mathcal{H}^*} < \E$ for $m, n \geq N$.  By the Riesz Representation Theorem, $\exists (y_n)_n \in \mathcal{H}$ such that for every $n$, $\phi_n(x) = \VEC{y_n}{x}\ \forall x \in \mathcal{H}$.  $(y_n)_n$ is Cauchy since given $\E > 0$ we can find $N$ such that $\norm{y_n - y_m}_{\mathcal{H}} = \norm{\phi_n - \phi_m}_{\mathcal{H}^*} < \E$.  Since $\mathcal{H}$ is a Hilbert space, then $(y_n)_n$ is convergent to some $y \in \mathcal{H}$.  By the Riesz Representation Theorem, $\exists \phi \in \mathcal{H}^*$ such that $\phi(x) = \VEC{y}{x}\ \forall x \in \mathcal{H}$.  Then $(\phi_n)_n$ converges to $\phi$ because $\norm{\phi_n - \phi}_{\mathcal{H}^*} = \norm{y_n - y}_\mathcal{H}$, which can be made arbitrary small by the definition of convergence.  Thus $\mathcal{H}^*$ is complete.  Also, $\VEC{\cdot}{\cdot}_{\mathcal{H}^*}$ is a well-defined inner product since $\VEC{\cdot}{\cdot}_\mathcal{H}$ is a well-defined inner product (i.e.~the properties of inner product hold).  Thus $\mathcal{H}^*$ is a Hilbert space.
\end{proof}









%%%%%%%%%%%%%%%%%%%%%%%%%%%%%%%%%%%%%%
\problem{Hunter and Nachtergaele 8.9}
\emph{Let $A \subset \mathcal{H}$ be such that $$\mathcal{M} = \left\{x \in \mathcal{H}\ |\ x \text{ is a finite linear combination of elements in $A$}\right\}$$ is a dense linear subspace of $\mathcal{H}$.  Prove that any bounded linear functional on $\mathcal{H}$ is uniquely determined by its values on $A$.  If $\{u_\alpha\}$ is an orthonormal basis, find a necessary and sufficient condition on a family of complex numbers $c_\alpha$ for there to be a bounded linear functional $\phi$ such that $\phi(u_\alpha) = c_\alpha$.} \\

\begin{proof}
    Suppose $\phi_1$ and $\phi_2$ are two bounded linear functionals such that $\phi_1(a) = \phi_2(a)$ for all $a \in A$.  Let $x \in \mathcal{H}$.  By denisty of $\mathcal{M}$, $\exists m_i \in \mathcal{M}$ such that $m_i \rightarrow x$.  By linearity of $\phi_1$ and $\phi_2$, $\phi_1(m_i) = \phi_2(m_i)$ for all $i = 1, 2, \dots$.  By the Riesz Representation Theorem, $\exists y_1, y_2$ such that $\phi_1(x) = \VEC{y_1}{x}$ and $\phi_2(x) = \VEC{y_2}{x}$ for all $x \in \mathcal{H}$.  Thus by continuity of inner products,
    \begin{align*}
        \phi_1(x) = \VEC{y_1}{x} = \lim_{i\rightarrow \infty}\VEC{y_1}{m_i} = \lim_{i\rightarrow \infty}\phi_1(m_i) = \lim_{i \rightarrow \infty}\phi_2(m_i) = \lim_{i \rightarrow \infty}\VEC{y_2}{m_i} = \VEC{y_2}{x} = \phi_2(x)
    \end{align*}
    Thus $\phi_1 \equiv \phi_2$, i.e.~bounded linear functionals are uniqueley determined by their values on $A$.
\end{proof}










%%%%%%%%%%%%%%%%%%%%%%%%%%%%%%%%%%%%%%
\problem{Hunter and Nachtergaele 8.11}
\emph{Prove that if $A\ :\ \mathcal{H} \rightarrow \mathcal{H}$ is a linear map and $\dim \mathcal{H} < \infty$, then $$\dim \ker A + \dim \ran A = \dim \mathcal{H}.$$  Prove that if $\dim \mathcal{H} < \infty$, then $\dim \ker A = \dim \ker A^*$.  In particular, $\ker A = \{0\}$ if and only if $\ker A^* = \{0\}$.} \\

\begin{proof}
    Let $\dim \mathcal{H} < \infty$ (say $\dim \mathcal{H} = n$).  Then $\dim \ker A < \infty$ and $\dim \ran A < \infty$ since $\ker A$ and $\ran A$ are subspaces of $\mathcal{H}$.  Let $\dim \ker A = m \leq n$ and let $\left\{u_1, u_2, \dots, u_m\right\}$ be a basis of $\ker A$.  Since $\ker A$ is a subspace of $\mathcal{H}$, this basis can be extended to a basis $\mathcal{U}$ of $\mathcal{H}$: $\mathcal{U} = \left\{u_1, u_2, \dots, u_m, v_{m+1}, v_{m+2}, \dots, v_{n}, \right\}$.  Let $x \in \mathcal{H}$.  Then
    \begin{align*}
        x &= a_1u_1 + a_2u_2 + \dots + a_mu_m + a_{m+1}v_{m+1} + a_{m+2}v_{m+2} + \dots + a_nv_n \\
        \implies Ax &= \cancelto{0}{a_1Au_1} + \cancelto{0}{a_2Au_2} + \dots + \cancelto{0}{a_mAu_m} + a_{m+1}Av_{m+1} + a_{m+2}Av_{m+2} + \dots + a_nAv_n \\
        &= a_{m+1}Av_{m+1} + a_{m+2}Av_{m+2} + \dots + a_nAv_n
    \end{align*}
    since $u_i \in \ker A$ for $i = 1, 2, \dots, m$.  Thus, $\left\{Av_{m+1}, Av_{m+2}, \dots, Av_n\right\}$ spans $\ran A$.  However it is also linearly independent since
    \begin{align*}
        c_{m+1}Av_{m+1} + c_{m+2}Av_{m+2} + \dots + c_nAv_n &= 0 \\
        \implies A\qty(c_{m+1}v_{m+1} + c_{m+2}v_{m+2} + \dots + c_nv_n) &= 0 \\
        \implies c_{m+1}v_{m+1} + c_{m+2}v_{m+2} + \dots + c_nv_n &\in \ker A \\
        \implies c_{m+1}v_{m+1} + c_{m+2}v_{m+2} + \dots + c_nv_n &= d_1u_1 + d_2u_2 + \dots + d_mu_m \qquad \text{for some $d_i \in \Cx$} \\
        \implies d_1 = d_2 = \dots = d_m = c_{m+1} = c_{m+2} = \dots = c_n &= 0 \qquad \text{since $\mathcal{U}$ is a basis}
    \end{align*}
    Thus $\left\{Av_{m+1}, Av_{m+2}, \dots, Av_n\right\}$ is linearly independent.  Since it also spans $\ran A$ then it is a basis of $\ran A$.  Thus $\dim \ran A = n - m$.  Thus, since $m + (n - m) = n$, then
    \begin{align*}
        \dim \ker A + \dim \ran A = \dim \mathcal{H}.
    \end{align*}

    Let $x \in \mathcal{H}$.  Then
    \begin{align*}
        x \in \ker A^* &\iff A^*x = 0 \\
        &\iff \VEC{y}{A^*x} = 0 \qquad \forall y \in \mathcal{H} \\
        &\iff \VEC{Ay}{x} = 0 \qquad \forall y \in \mathcal{H} \\
        &\iff x \perp Ay \qquad \forall y \in \mathcal{H} \\
        &\iff x \perp (\ran A) \\
        &\iff x \in (\ran A)^\perp
    \end{align*}
    Thus $\ker A^* = (\ran A)^\perp$ and $\dim \ker A^* = \dim (\ran A)^\perp$.  However since $\dim \ker A + \dim \ran A = \dim \mathcal{H}$ and $\dim \ran A + \dim (\ran A)^\perp = \dim \mathcal{H}$, then $\dim (\ran A)^\perp = \dim \ker A$.  Thus,
    \begin{align*}
        \dim \ker A = \dim \ker A^*.
    \end{align*}
\end{proof}










%%%%%%%%%%%%%%%%%%%%%%%%%%%%%%%%%%%%%%
\problem{Hunter and Nachtergaele 8.12}
\emph{Suppose that $A\ :\ \mathcal{H} \rightarrow \mathcal{H}$ is a bounded, self-adjoint linear operator such that there is a constant $c > 0$ with $$c\norm{x} \leq \norm{Ax} \qquad \text{for all } x \in \mathcal{H}.$$  Prove that there is a unique solution $x$ of the equation $Ax = y$ for every $y \in \mathcal{H}$.}



\end{document}
