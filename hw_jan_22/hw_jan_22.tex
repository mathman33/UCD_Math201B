\documentclass[12pt]{article}

\usepackage{amssymb, amsmath, amsfonts}
\usepackage{moreverb}
\usepackage{graphicx}
\usepackage{enumerate}
\usepackage[margin=0.75in]{geometry}
\usepackage{graphics}
\usepackage{color}
\usepackage{array}
\usepackage{float}
\usepackage{hyperref}
\usepackage{textcomp}
\usepackage{bbold}
\usepackage{alltt}
\usepackage{physics}
\usepackage{mathtools}
\usepackage{amsthm}
\usepackage{tikz}
\usetikzlibrary{positioning}
\usetikzlibrary{arrows}
\usepackage{pgfplots}
\usepackage{bigints}
\allowdisplaybreaks
\pgfplotsset{compat=1.12}

\theoremstyle{plain}
\newtheorem*{theorem*}{Theorem}
\newtheorem{theorem}{Theorem}
\newtheorem*{lemma*}{Lemma}
\newtheorem{lemma}{Lemma}

\newenvironment{definition}[1][Definition]{\begin{trivlist}
\item[\hskip \labelsep {\bfseries #1}]}{\end{trivlist}}

\title{\bf HW \#2}
\author{\bf Sam Fleischer}
\date{\bf January 22, 2015}

\pgfplotsset{compat=1.12}

\begin{document}
\textbf{MATH 201B \hfill Analysis \ \ \hfill Winter 2016\ \ \ }

{\let\newpage\relax\maketitle}

\section*{Exercise 1.9}
\textit{Verify the linearity of the integral as given in 1.5(7) by completing the steps outlined below.  In what follows, $f$ and $g$ are nonnegative summable functions.}

\begin{definition}[Definition: Simple Function]
    Let $(\Omega, \Sigma, \mu)$ be a measure space.  A \emph{simple function} is a function $\phi\ :\ \Omega \rightarrow [0, \infty)$ given by
    \begin{align*}
        \phi = \sum_{i=1}^N c_i \mathcal{X}_{E_i}
    \end{align*}
    where $c_i \geq 0$, $E_i \in \Sigma$ for $i = 1, \dots, N$, and $E_i \cap E_j = \emptyset$ for $i \neq j$.  The integral of a simple function is given by
    \begin{align*}
        \int_\Omega \phi \dd \mu = \sum_{i = 1}^N c_i \mu(E_i)
    \end{align*}
    Define $\mathcal{S}_\Omega$ as the set of all simple functions on $\Omega$, i.e.
    \begin{align*}
        \mathcal{S}_\Omega = \{\phi \in [\Omega \rightarrow [0, \infty)]\ :\ \phi \text{ \rm is a simple function }\}
    \end{align*}
\end{definition}
\begin{definition}[Definition 1: Lebesgue Integral]
    Let $f\ :\ \Omega \rightarrow [0, \infty)$ be a positive, measurable real-valued function on a measure space $(\Omega, \Sigma, \mu)$.  Then
    \begin{align*}
        \int_\Omega f \dd \mu = \int_0^\infty F_f(t) \dd t
    \end{align*}
    where $F_f(t) = \mu(f^{-1}(t, \infty))$.
\end{definition}
\begin{definition}[Definition 2: Lebesgue Integral]
    Let $f\ :\ \Omega \rightarrow [0, \infty)$ be a positive, measurable real-valued function on a measure space $(\Omega, \Sigma, \mu)$.  Then
    \begin{align*}
        \int_\Omega f \dd \mu = \sup_{\substack{0\leq\phi\leq f \\ \phi \in \mathcal{S}_\Omega}} \left\{\int_\Omega \phi \dd \mu\right\}
    \end{align*}
\end{definition}
\begin{lemma}[Equivalence of Definitions for Simple Functions]
    Let $\Phi$ be a simple function on a measure space $(\Omega, \Sigma, \mu)$.  Then
    \begin{align*}
        \int_0^\infty F_\Phi(t) \dd t = \sup_{\substack{0 \leq \phi \leq \Phi \\[.05cm] \phi \in \mathcal{S}_\Omega}} \left\{\int_\Omega \phi \dd \mu\right\}
    \end{align*}
    where $F_\Phi(t) = \mu(\Phi^{-1}(t, \infty))$.  In other words, the two definitions for Lebesgue integral are equivalent for simple functions.
\end{lemma}
\begin{proof}
    Since $\Phi$ is a simple function, then
    \begin{align*}
        \sup_{\substack{0 \leq \phi \leq \Phi \\[.05cm] \phi \in \mathcal{S}_\Omega}}\left\{\int_\Omega \phi \dd \mu\right\} = \int_\Omega \Phi \dd \mu = \sum_{i = 1}^N c_i \mu(E_i)
    \end{align*}
    where $c_i$ and $E_i$, $i = 1, \dots, N$ are defined for $\Phi$.  Since $E_i \cap  E_j = \emptyset$ for $i \neq j$, then the maximum of $\Phi$ is the maximum of the coefficients $\{c_i\}$, denoted $c_\Phi$.
    \begin{align*}
        c_\Phi = \max_{x \in \Omega} \Phi(x) = \max_{1 \leq i \leq N} \{c_i\}
    \end{align*}
    This shows that $\Phi^{-1}(t, \infty) = \emptyset$ for $t \geq c_\Phi$, and thus
    \begin{align*}
        \int_0^\infty F_\Phi(t) \dd t = \int_0^{c_\Phi} F_\Phi(t) \dd t
    \end{align*}
    We can construct a set $\{d_i\}$ such that $\{d_i\} = \{c_i\}$ but $d_1 \leq d_2 \leq \dots \leq d_N = c_\Phi$.  In other words, the set $\{d_i\}$ is simply the set $\{c_i\}$ ordered from least to greatest.  Then
    \begin{align*}
        \int_0^{c_\Phi} F_\Phi(t) \dd t = \int_0^{d_1} F_\Phi(t) \dd t + \int_{d_1}^{d_2} F_\Phi(t) \dd t + \dots + \int_{d_{N-1}}^{d_N} F_\Phi(t) \dd t = \sum_{k = 1}^N \int_{d_{k - 1}}^{d_k} F_\Phi(t) \dd t
    \end{align*}
    where, for ease, we define $d_0 = 0$.  We can easily form the set $\{D_i\}$ such that $D_i$ corresponds to $d_i$.  In other words, we can write $\int_\Omega \Phi \dd \mu = \sum_{i=1}^N d_i \mu(D_i)$.  Note that for $t \in (0, d_1)$ we have
    \begin{align*}
        F_\Phi(t) = \mu\qty(\bigcup_{i=1}^ND_i) = \sum_{i=1}^N \mu(D_i)
    \end{align*}
    In general, for $t \in (d_{k-1}, d_k)$ we have
    \begin{align*}
        F_\Phi(t) = \mu\qty(\bigcup_{i=k}^N D_i) = \sum_{i=k}^N \mu(D_i)
    \end{align*}
    Thus,
    \begin{align*}
        \int_0^{c_\Phi} F_\Phi(t) \dd t &= \sum_{k = 1}^N \int_{d_{k - 1}}^{d_k} F_\Phi(t) \dd t \\
        &= \sum_{k=1}^N\int_{d_{k-1}}^{d_k}\sum_{i=k}^N \mu(D_i) \dd t \\
        &= \sum_{k=1}^N \sum_{i=k}^N \qty[\mu(D_i) \int_{d_{k-1}}^{d_k} \dd t ]\ \ \ \ \ \text{by linearity of Riemann integrals}\\
        &= \sum_{k=1}^N \sum_{i=k}^N \qty[\mu(D_i)\qty(d_k - d_{k-1})] \\
        &= \sum_{k=1}^N \qty(d_k - d_{k-1}) \sum_{i=k}^N \mu(D_i) \\
        &= (d_1  - 0)\qty[\mu(D_1) + \dots + \mu(D_N)] + (d_2 - d_1)\qty[\mu(D_2) + \dots + \mu(D_N)] + \dots \\
        &\ \ \ \dots + (d_{N-1} - d_{N-2})\qty[\mu(D_{N-1}) + \mu(D_N)] + (d_N - d_{N-1})\mu(D_N) \\
        &= d_1\mu(D_1) + d_2\mu(D_2) + \dots + d_N\mu(D_N)\ \ \ \ \ \text{by combining like-terms} \\
        &= \sum_{i=1}^N c_i \mu(E_i)
    \end{align*}
    which completes the proof.
\end{proof}
\begin{lemma}[Equivalence of Definitions for Arbitrary Functions]
    Let $f$ be a measurable function on a measure space $(\Omega, \Sigma, \mu)$.  Then
    \begin{align*}
        \int_0^\infty F_f(t) \dd t = \sup_{\substack{0 \leq \phi \leq f \\[.05cm] \phi \in \mathcal{S}_\Omega}} \left\{\int_\Omega \phi \dd \mu\right\}
    \end{align*}
    where $F_f(t) = \mu(f^{-1}(t, \infty))$.  In other words, the two definitions for Lebesgue integral are equivalent.
\end{lemma}
\begin{proof}
    Consider the set $E_{n,k} = \left\{x\ :\ f(x) > \frac{k}{2^n}\right\}$ for $n \in \{1, 2, \dots\}$ and $k \in \{1, 2, \dots, 4^n\}$.  Define $\{\phi_n\}_{n=1}^\infty$ by
    \begin{align*}
        \phi_n = \frac{1}{2^n}\sum_{k=1}^{4^n}\mathcal{X}_{E_{n,k}}
    \end{align*}
    Note that $E_{n,a} \subset E_{n,b}$ if $a > b$.  Then note that for any $x$, either $x \in E_{n, 4^n}$ or $\exists\ell$ such that $x \in E_{n, \ell}$ but $x \not\in E_{n, \ell+1}$.

    If $x \in E_{n, 4^n}$, then by its definition, $f(x) > \frac{4^n}{2^n} = 2^n$ and $\phi_n(x) = \frac{1}{2^n}\sum_{k=1}^{4^n} 1 = 2^n$.  If $x \in E_{n,\ell}$ but $x \not \in E_{n,\ell+1}$, then by its definition $\frac{\ell + 1}{2^n} > f(x) > \frac{\ell}{2^n}$, but $\phi_n(x) = \frac{\ell}{2^n}$.  In either case, $\phi_n \leq f$ for each $n = 1, 2, 3, \dots$.

    Next we show $\phi_{n+1} \geq \phi_n$.  Suppose $f(x) < 2^n$.  Then $\exists \ell \in \{1, 2, \dots, 4^n\}$ such that $\frac{\ell}{2^n} \leq f(x) < \frac{\ell + 1}{2^n}$.  Then either $\frac{2\ell}{2^{n+1}} < \frac{2\ell + 1}{2^{n+1}} \leq f(x) < \frac{2\ell + 2}{2^{n+1}}$ (in which case $\phi_{n+1}(x) = \frac{2\ell + 1}{2^{n+1}} > \phi_n$) or $\frac{2\ell}{2^{n+1}} \leq f(x) \leq \frac{2\ell + 1}{2^{n+1}} < \frac{2\ell + 2}{2^{n+1}}$ (in which case $\phi_{n+1}(x) = \frac{2\ell}{2^{n+1}} = \phi_n(x)$).

    Suppose $f(x) = 2^n$.  Then $\phi_{n+1}(x) = 2^n = \phi_n(x) = f(x)$.

    Suppose $f(x) > 2^n = \phi_n(x)$.  Then either $f(x) \geq 2^n + \frac{1}{2^{n+1}}$ or $f(x) < 2^n + \frac{1}{2^{n+1}}$.  If $f(x) \geq 2^n + \frac{1}{2^{n+1}}$, the $\phi_{n+1}(x) \geq 2^n + \frac{1}{2^{n+1}} > \phi_n(x)$.  If $f(x) < 2^n + \frac{1}{2^{n+1}}$, then $\phi_{n+1}(x) = 2^n = \phi_n(x)$.

    In all cases, $\phi_{n+1}(x) \geq \phi_n(x)$ for all $x$ and $n$.  Thus, we that \emph{(i)} $f$ is the pointwise limit of $\phi_n$, and \emph{(ii)} $\{\phi_n\}_n$ is a non-decreasing sequence of functions.  These are the two hypotheses of the monotone convergence theorem, and so
    \begin{align*}
        \int_\Omega f \dd \mu = \lim_{n\rightarrow \infty} \int_\Omega \phi_n \dd \mu
    \end{align*}
    where the above integrals are defined using the second defintion of Lebesgue integrals.

    Now note that
    \begin{align*}
        f^{-1}(t, \infty) = \bigcup_{n=1}^\infty\{\phi_n^{-1}(t, \infty)\}
    \end{align*}
    since $f > \phi_n$ for all $n$ and $\phi_n \rightarrow f$, and
    \begin{align*}
        \bigcup_{n=1}^\infty\{\phi_n^{-1}(t, \infty)\} = \lim_{n\rightarrow \infty} \phi_n^{-1}(t, \infty)
    \end{align*}
    since $\{\phi_n\}_n$ is an increasing function.  Thus,
    \begin{align*}
        F_f(t) = \mu(f^{-1}(t, \infty)) = \mu(\lim_{n\rightarrow \infty} \phi_n^{-1}(t, \infty)) = \lim_{n\rightarrow \infty}F_{\phi_n}(t, \infty)
    \end{align*}
    However, $\{F_{\phi_n}(t, \infty)\}_n$ is monotone increasing to $F_f(t, \infty) \in \mathbb{R}$, and thus again by the monotone convergence theorem,
    \begin{align*}
        \int_0^\infty F_f(t) \dd t = \lim_{n\rightarrow \infty}\int_0^\infty F_{\phi_n}(t)\dd t
    \end{align*}
    By Lemma 1, the two definitions of Lebesgue integration are equivalent for simple functions, and thus
    \begin{align*}
        \int_0^\infty F_{\phi_n}(t) \dd t = \int_\Omega \phi_n(t)
    \end{align*}
    for all $n$.  Then by the result above,
    \begin{align*}
        \lim_{n\rightarrow \infty}\qty[\int_0^\infty F_{\phi_n}(t) \dd t] &= \lim_{n\rightarrow \infty}\qty[\int_\Omega \phi_n(t)] \\
        \implies \int_0^\infty F_f(t) \dd t &= \sup_{\substack{0 \leq \phi\leq f \\ \phi \in S_\Omega}}\left\{\int_\Omega \phi \dd \mu\right\}
    \end{align*}
\end{proof}

\subsubsection*{ a)}
\textit{Show that $f + g$ is also summable.  In fact, by a simple argument $\int(f + g) \leq 2\qty(\int f + \int g)$.} \\

Suppose $f$ and $g$ are summable.  Thus
\begin{align*}
    \int f < \infty\ \ \ \ \ \ \text{and}\ \ \ \ \ \ \int g < \infty
\end{align*}
By the simple function definition of Lebesgue integrals,
\begin{align*}
    \sup_{\substack{0 \leq \phi \leq f \\ \phi \in S_\Omega}}\left\{\int \phi \right\} < \infty\ \ \ \ \ \ \ \text{and}\ \ \ \ \ \ \sup_{\substack{0 \leq \psi \leq g \\ \psi \in S_\Omega}}\left\{\int \psi \right\} <  \infty
\end{align*}
Thus construct two sequences of simple functions $\{\phi_n\}_n$ and $\{\psi_n\}_n$ such that $\int \phi_n \rightarrow \int f$ and $\int \psi_n \rightarrow \int g$.  Then choose any arbitrary $\varepsilon$.  Then
\begin{align*}
    \exists N_\phi, N_\psi\ \ \text{ such that if }\ \ n \geq \max\{N_\phi, N_\psi\},\ \ \text{ then }\ \ \int f - \int \phi_n < \frac{\varepsilon}{2}\ \ \text{ and }\ \ \int g - \int \psi_n < \frac{\varepsilon}{2}
\end{align*}

By the definition of the integration of simple functions, there are disjoint sets $\{E_i\}_{i = 1}^{N_E}$ and $\{F_j\}_{j = 1}^{N_F}$ such that
\begin{align*}
    \int \phi_n = \sum_{i = 1}^{N_E} c_i\mathcal{X}_{E_i}\ \ \ \ \ \ \text{and}\ \ \ \ \ \ \int \psi_n = \sum_{j = 1}^{N_F} d_j\mathcal{X}_{F_j}
\end{align*}
However we can construct the set $\{G_{i,j}\} = \{E_i \cap F_j\ :\ 1 \leq i \leq N_E, 1 \leq j \leq N_F\}$, and thus
\begin{align*}
    \int \phi_n = \sum_{i = 1}^{N_E}\sum_{j = 1}^{N_F} c_i\mathcal{X}_{G_{i,j}}\ \ \ \ \ \ \text{and}\ \ \ \ \ \ \int \psi_n = \sum_{j = 1}^{N_F}\sum_{i = 1}^{N_E} d_j\mathcal{X}_{G_{i,j}}
\end{align*}
By linearity of finite sums,
{\color{red} omg finish up}

% To show $\int (f + g) \leq 2\qty(\int f + \int g)$, first note that
% \begin{align*}
%     S_{f + g}(t) = \{x\ :\ (f+g)(x) > t\} \subset \left\{x\ :\ f(x) > \frac{t}{2}\right\} \cup \left\{x\ :\ g(x) > \frac{t}{2}\right\} = S_f\qty(\frac{t}{2}) \cup S_g\qty(\frac{t}{2})
% \end{align*}
% Since if $f(x) \leq \frac{t}{2}$ and $g(x) \leq \frac{t}{2}$ then $(f + g)(x) = f(x) + g(x) \leq t$.  By properties of measures, 
% \begin{align*}
%     \mu(S_{f + g}(t)) \leq \mu\qty(S_f\qty(\frac{t}{2}) \cup S_g\qty(\frac{t}{2})) &\leq \mu\qty(S_f\qty(\frac{t}{2})) + \mu\qty(S_g\qty(\frac{t}{2})) \\
%     \implies \int_0^\infty \mu(S_{f + g}(t)) \dd t &\leq \int_0^\infty \mu\qty(S_f\qty(\frac{t}{2}))\dd t + \int_0^\infty \mu\qty(S_g\qty(\frac{t}{2}))\dd t
% \end{align*}
% Note the integral on the right hand side can split linearly because it is a Riemann integral.  By $u$-substitution with $u = \frac{t}{2}$, we get
% \begin{align*}
%     \int_0^\infty \mu(S_{f + g}(t))\dd t \leq 2\int_0^\infty S_f(t)\dd t + 2\int_0^\infty S_g(t)\dd t
% \end{align*}
% Note the constant $2$ can be factored of each integral on the right hand side linearly because they are Riemann integrals.  Thus, by definition,
% \begin{align*}
%     \int (f + g) \leq 2\qty(\int f + \int g)
% \end{align*}
% and since $f$ and $g$ are summable, $\int f$ and $\int g$ are finite, which proves $\int (f + g)$ is finite, i.e.~$f + g$ is summable. \hfill $\square$

\subsubsection*{ b)}
\textit{For any integer $N$ find two functions $f_N$ and $g_N$ that take only finitely many values, such that $|\int f - \inf f_N| \leq \frac{C}{N}$, $|\int g - \int g_N| \leq \frac{C}{N}$, $|int(f + g) - \int(f_N - g_N)| \leq \frac{C}{N}$ for some constant $C$ independent of $N$.}

\subsubsection*{ c)}
\textit{Show that for $f_N$ and $g_N$ as above $\int(f_N + g_N) = \int f_N + \int g_N$, thus proving the addivitivity of te integral for nonnegative functions.}

\subsubsection*{ d)}
\textit{In a similar fashion, show that for $f, g \geq 0$, $\int(f - g) = \int f - \int g$.}

\subsubsection*{ e)}
\textit{Now use c) and d) to prove the linearity of the integral.}

\section*{Exercise 1.12}
\textit{Find a simple condition for $f_n(x)$ so that}
\begin{align*}
    \sum_{n=0}^\infty \int_\Omega f_n(x) \mu(\dd x) = \int_\Omega \qty[\sum_{n=0}^\infty f_n(x)]\mu(\dd x)
\end{align*}

\section*{Exercise 1.13}
\textit{Let $f$ be the function on $\mathbb{R}^n$ defined by $f(x) = |x|^{-p}\mathcal{X}_{\{|x| < 1\}}(x)$.  Compute $\int f \dd \mathcal{L}^n$ in two ways: (i) Use polar coordinates and compute the integral by the standard calculus method.  (ii) Compute $\mathcal{L}^n\qty(\{x\ :\ f(x) > a\})$ and then use Lebesgue's definition.}
\begin{enumerate}[(i)]
    \item
        First note that
        \begin{align*}
            f(x) =
            \begin{cases} 
                |x|^{-p} & \text{if $|x| < 1$} \\
                0 & \text{else}
            \end{cases}
        \end{align*}
        Then note that polar coordinates on $\mathbb{R}^n$ are $(r, \phi, \theta_1, \theta_2, \dots, \theta_{n-2})$ where $r \in [0, \infty)$, $\phi \in [0, 2\pi)$, and $\theta_i \in [0, \pi)$ for $i = 1, 2, \dots, n - 2$.
        \begin{align*}
            \int f \dd \mathcal{L}^n &= \int_0^\pi \int_0^\pi \dots \int_0^\pi \int_0^{2\pi} \int_0^\infty r^{-p}\ \dd r\ \dd \phi\ \dd \theta_1 \dots\ \dd \theta_{n-3}\ \dd \theta_{n-2}
        \end{align*}
        We can use Fubini's theorem since each of these integrals are Riemann integrals.  Thus,
        \begin{align*}
            \int f \dd \mathcal{L}^n = 2\pi^{n-1}\int_0^\infty r^{-p} \dd r = 2\pi^{n-1}\int_0^1 r^{-p} \dd r
        \end{align*}
        since we know $f(x) = 0$ whenever $r = |x| \geq 1$.  This integral is dependent on $p$ in the following way:
        \begin{align*}
            \int f \dd \mathcal{L}^n = \begin{cases}
                2\pi^{n-1}\frac{1}{1 - p} & \text{if $p < 1$} \\
                +\infty & \text{if $p \geq 1$}
            \end{cases}
        \end{align*}
    \item
        If $0 < p < 1$, $f$ is a decreasing function of modulus and $f\rightarrow\infty$ as $x\rightarrow 0$.  If $p < 0$, $f$ is an increasing function of modulus and $f \rightarrow \infty$ as $|x|\rightarrow 1$.  Thus it should be intuitive that $f^{-1}(a, \infty)$ is either a smaller $n$-sphere if $0 < p < 1$ or a shell of an $n$-sphere if $p < 0$.
        \begin{align*}
            \mathcal{L}^n(\{x\ :\ f(x) > a\}) &= \mathcal{L}^n(\{x \in B_1(0)\ :\ |x|^{-p} > a\}) \\
            &= \begin{cases}
                \mathcal{L}^n(\{x\ \in B_1(0)\ :\ |x| < a^{-\frac{1}{p}}\}) & \text{if $0 < p < 1$} \\
                \mathcal{L}^n(\{x\ \in B_1(0)\ :\ |x| > a^{-\frac{1}{p}}\}) & \text{if $p < 0$}
            \end{cases} \\
            &= \begin{cases}
                \mathcal{L}^n(B_{a^{-\frac{1}{p}}}(0)) & \text{if $0 < p < 1$} \\
                \mathcal{L}^n(B_1(0)) - \mathcal{L}^n(B_{a^{-\frac{1}{p}}}(0)) & \text{if $p < 0$}
            \end{cases}
        \end{align*}
        But the Lebesgue measures of balls are relatively simple to compute:
        \begin{align*}
            \mathcal{L}^n(B_r(x)) = \frac{2\pi^{\frac{n}{2}}r^n}{n\Gamma(\frac{n}{2})}
        \end{align*}
        Thus,
        \begin{align*}
            \mathcal{L}^n(\{x\ :\ f(x) > a\}) &= \begin{cases}
                \dfrac{2\pi^{\frac{n}{2}}a^{-\frac{n}{p}}}{n\Gamma(\frac{n}{2})} & \text{if $0 < p < 1$} \\
                \dfrac{2\pi^{\frac{n}{2}}}{n\Gamma(\frac{n}{2})} - \dfrac{2\pi^{\frac{n}{2}}a^{-\frac{n}{p}}}{n\Gamma(\frac{n}{2})} & \text{if $p < 0$}
            \end{cases} \\
            &= \begin{cases}
                \dfrac{2\pi^{\frac{n}{2}}a^{-\frac{n}{p}}}{n\Gamma(\frac{n}{2})} & \text{if $0 < p < 1$} \\
                \dfrac{2\pi^{\frac{n}{2}}}{n\Gamma(\frac{n}{2})}\qty(1 - a^{-\frac{n}{p}}) & \text{if $p < 0$}
            \end{cases}
        \end{align*}
\end{enumerate}

\end{document}
